\chapter{Conclusion}
Hydrogels are being increasingly used in biomedical applications. It has therefore become important to understand their characteristics. In order to understand the suitability of hydrogels to a specific application, it is convenient to be able to regenerate the material response in a simulation with the help of a material model. Various mathematical models have been developed over the years to describe the characteristics of hydrogels. However, these have been based on the underlying physical and chemical phenomena. Since, hydrogels also exhibit viscoelastic properties, it was sought to characterize the the hydrogels using the finite viscoelasticity theory in this work. For this, user material (\texttt{UMAT}) subroutines for finite viscoelasticity theory using the Ogden class of strain energy functions with one, two and three relaxation mechanisms were implemented. The necessary theory required to implement the material model as well as the preliminary theory of continuum mechanics was briefly summarized. Furthermore, the step-by-step implementation details for the \texttt{UMAT} were also explained. 

Thereafter, so that the implemented material model could characterize a hydrogel material, parameter identification was carried out using a non-linear least square optimization procedure along with a twin implementation of the \texttt{UMAT} in MATLAB. Here it was found that, the \texttt{UMAT} with a single relaxation mechanism was not able to replicate the material response of the hydrogel material from the experimental data. Additionally, although the optimal set parameters for \texttt{UMAT}s with two and three relaxation mechanisms were identified, it was found later on, that the models with these parameters generated a material response in ABAQUS simulations which disagreed with the experimental data. However, this does not altogether disregard the use of the finite viscoelasticity theory to characterize hydrogel materials, but suggests towards finding another approach for the parameter identification procedure. For future work in this regard, as an alternate, it could be possible to use the material response from the finite element simulation in ABAQUS for calculation of the cost function in the optimization procedure. Also, for material models that employ Ogden class of strain energy functions, it is recommended to perform the parameter identification procedure using experimental data from additional load cases, viz. biaxial tension and pure shear. Lastly, as an improvement, the \texttt{UMAT} could also be optimized to reduce the number of memory allocations for local variables in the models with two or more relaxation mechanisms to make the subroutine faster and efficient.



