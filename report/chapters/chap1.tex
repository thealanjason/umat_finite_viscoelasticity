\chapter{Introduction}

Hydrogel materials are very soft materials consisting of polymer networks and solvent molecules. These materials may exhibit large volume changes depending on their external chemical and mechanical environment. They also have viscoelastic properties which is common for many polymeric materials. Due to their favourable properties, they are increasingly used in biomedical applications. For example, hydrogels are used for contact eye lenses and also as filler materials for intervertebral spinal discs. In order to use hydrogels in an effective manner for such applications, it has become important to characterize these materials with mathematical models and to thoroughly understand their properties. These mathematical (material) models are then used in simulations to design solutions to various problems encountered in the field. 

Previously, material models for hydrogels, that took into account the micro-scale chemical and mechanical phenomena, have been developed\cite{Long2014Oct}. However, at the macro-scale, the mechanical response of an hydrogel is similar to that of a viscoelastic material. Hence, the goal of this work is to characterize the macro-scale mechanical properties of a hydrogel by using only the finite viscoelasticity theory as described in \cref{chapter:two}. For this a user material subroutine \texttt{UMAT} for the material model will be implemented, details of which are given in \cref{chapter:three}, so that it can be used for finite element simulations in ABAQUS. Furthermore, the parameter identification procedure for the implemented viscoelastic material model will also be explored in \cref{chapter:four} followed by numerical simulations in \cref{chapter:five}. 




