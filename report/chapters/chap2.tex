\chapter{Theory}%
\label{chapter:two}

\section{Preliminary Continuum Mechanics}
In this section, the necessary concepts and results of Continuum Mechanics theory will be discussed, without getting into the details. The contents are largely based on the lecture notes provided by Prof. Itskov for his course on Continuum Mechanics\cite{ContinuumMechanics2020} at RWTH Aachen University. For an in-depth coverage of these fundamentals, attending lectures or reading any reference book on this subject, if not required, is highly recommended. The following concepts and results are instrumental to understanding various material modelling theories, and hence, are quickly reviewed.

\subsection{Kinematics}

\paragraph*{Material Bodies and Configurations}%
A \emph{material body}, say B, is understood to be a set of material particles occupying some bounded region, say \(\mathcal{B}\), of a three-dimensional Euclidian space at various times. These particles are characterized by some positive measure called mass. One of the basic assumptions of continuum mechanics is that the material is continuously distributed in bodies at all scales. Thus, only the mathematical treatment of this bounded region is carried out without paying much attention to the actual physical body. Mathematically, a continuum is defined as a continuous compact metric space, while for practical purposes we are describing smooth solids or confined liquids.

\begin{figure}[htpb]
    \centering
    \includegraphics[width=\textwidth]{body_motion.pdf}
    \caption{Motion of a body}%
    \label{fig:body_motion}
\end{figure}

At any time \(t\), a mapping of the body B into a bounded region \(\mathcal{B}\) of the three-dimensional Euclidian space is assumed to be one to one and is referred to as a \emph{configuration}. Thus any point P of the body B can be associated with a position vector \(\bm{x} \in \mathbb{E}^{3}\). To describe the motion and deformation of the body B it is convenient to fix some configuration \(\mathcal{B}_{0}\) at time \(t_0\). This fixed configuration \(\mathcal{B}_{0}\) is called the \emph{reference configuration}. 

\paragraph*{Coordinates, Position Vectors and Displacement} With a coordinate system \(\theta^{i} (i = 1,2,3)\) the point P can be defined by its coordinates at time \(t_{0}\). Thus we can write
\begin{equation}
    \bm{x} = \bm{x}(\theta^{1}, \theta^{2}, \theta^{3}, t), \quad
     \theta^{i}=\theta^{i}(\bm{x}, t), \quad 
     i = 1,2,3
\end{equation}
and
\begin{equation}
    \bm{X} = \bm{x}(\theta^{1}, \theta^{2}, \theta^{3}, t_{0}) = \bm{X}(\theta^{1}, \theta^{2}, \theta^{3}), \quad \theta^{i}=\theta^{i}(\bm{X}, t_{0}), \quad i = 1,2,3
\end{equation}

where \(\bm{X}\) denotes the \emph{position vector} of the point \(\mathrm{P}_{0}\) in the reference configuration (\cref{fig:body_motion}). The coordinates \(\theta^{i}(i = 1,2,3)\) identify the point P at any time. They are often referred to as convective coordinates because the coordinate lines are embedded into the material and deform with the body.

The difference between the position vectors of the point P in the current and reference configuration 
\begin{equation}
    \bm{u} = \bm{u}(\theta^{1}, \theta^{2}, \theta^{3}, t) = \bm{x} - \bm{X}
\end{equation}
is called displacement.

The Euclidian space is characterized by the existence of an orthonormal basis given by a set of mutually orthogonal unit vectors, say \(\bm{e}_{i}\), with the property \(\bm{e}_{i} \cdot \bm{e}_{i} = \delta^{ij}\), where \(\delta^{ij}\) denotes the Kronecker delta and is defined by 
\begin{equation}
    \delta_{ij} =\delta^{ij} = \delta^{i}_{j} = \begin{cases}
        1,              & \text{for } i = j\\
        0,              & \text{for } i \neq j
    \end{cases}
\end{equation}

With respect to this basis, the position vectors and displacement can be expressed as
\begin{align}
\bm{X} &= X^{i} \bm{e}_{i}, \quad &X^{j} &= \bm{X} \cdot \bm{e}_j,  &j = 1,2,3 \\
\bm{u} &= u^{i} \bm{e}_{i}, \quad &u^{j} &= \bm{u} \cdot \bm{e}_j,  &j = 1,2,3 \\
\bm{x} &= x^{i} \bm{e}_{i}, \quad &x^{j} &= \bm{x} \cdot \bm{e}_j = X^{j} + u^{j}, &j = 1,2,3 
\end{align}
where henceforth Einstein's summation convention over repeated indexes is applied. \(X^{j}\) and \(x^{j} (j = 1,2,3)\) are called referential and current coordinate, respectively.

Other bases suitable for the description of deformation are formed by the vectors tangent to the coordinate lines in the reference and current cofigurations as shown in \cref{fig:body_motion}. These tangent vectors are defined by the derivatives of the position vectors with respect to the convective coordinates. For short, these derivatives will be denoted by
\(
    \dfrac{\partial \left( \bullet \right) }{\partial \theta^{i}}=\left( \bullet \right)_{,i}
\)
.Thus
\begin{equation}
\bm{G}_{i}=\dfrac{\partial \bm{X}}{\partial \theta ^{i}}=\bm{X}_{,i}={X^{j}}_{,i}\:{e_{j}}, \quad  
\bm{g}_{i}=\dfrac{\partial \bm{x}}{\partial \theta ^{i}}=x_{,i}={\bm{x}^{j}}_{,i}\:{e_{j}}, \quad 
i = 1,2,3
\end{equation}
where and henceforth we assume that the functions \(\bm{X} = \bm{X}(\theta^{1}, \theta^{2}, \theta^{3})\) and \(\bm{x} = \bm{x}(\theta^{1}, \theta^{2}, \theta^{3}, t)\) are sufficiently differentiable.

It is also useful to define the bases dual to \(\bm{G}_i\) and \(\bm{g}_i (i = 1,2,3)\):
\begin{equation}
    \bm{G}_{i} \cdot \bm{G}^{j} = \delta_{i}^{j}, \quad
    \bm{g}_{i} \cdot \bm{g}^{j} = \delta_{i}^{j}, \quad
    i,j = 1,2,3.
\end{equation}

Thus we can write
\begin{equation}
    \bm{G}^i = \dfrac{\partial \theta^{i}}{\partial X^{j}}\bm{e}^{j}, \quad
    \bm{g}^i = \dfrac{\partial \theta^{i}}{\partial x^{j}}\bm{e}^{j}, \quad
    i,j = 1,2,3.
\end{equation}

\paragraph*{Deformation Gradient}
The current postion \(\bm{x}\) of some point P and its displacement \(\bm{u}\) characterize only the \emph{motion} of the body in this point. In order to describe its deformation it is necessary to know how fast these vectors change in a neighbourhood of this point. For example, there is no deformation if all points get the same displacement. In this case we deal with the rigid body motion.

The deformation gradient is defined as the gradient of the position vector in the current configuration and is denoted by \(\mathbf{F}\):
\begin{equation}
    \mathbf{F} = \text{Grad}\,{\bm{x}} = 
    \text{F}^{i}_{.j}\,\bm{e}_{i} \otimes \bm{e}^{j}
\end{equation}
where the matrix \(\left[\text{F}^{i}_{.j}\right]\) is given by 

\begin{equation}
    \left[\text{F}^{i}_{.j}\right]\ = 
    \begin{bmatrix}
        \pdv{x^{1}}{X^{1}} & \pdv{x^{1}}{X^{2}} & \pdv{x^{1}}{X^{2}} \\[0.5em]
        \pdv{x^{2}}{X^{1}} & \pdv{x^{2}}{X^{2}} & \pdv{x^{2}}{X^{2}} \\[0.5em]
        \pdv{x^{3}}{X^{1}} & \pdv{x^{3}}{X^{2}} & \pdv{x^{3}}{X^{2}}
    \end{bmatrix} = 
    \begin{bmatrix}
        \pdv{u^{1}}{X^{1}} + 1 & \pdv{u^{1}}{X^{2}} & \pdv{u^{1}}{X^{2}} \\[0.5em]
        \pdv{u^{2}}{X^{1}} & \pdv{u^{2}}{X^{2}} + 1 & \pdv{u^{2}}{X^{2}} \\[0.5em]
        \pdv{u^{3}}{X^{1}} & \pdv{u^{3}}{X^{2}} & \pdv{u^{3}}{X^{2}} + 1
    \end{bmatrix}
\end{equation}

The deformation gradient can also be written in a more compact form using the tangent vectors.
\begin{equation}
    \mathbf{F} = \bm{g}_{i} \otimes \bm{G}^{j}
\end{equation}
\paragraph*{Deformation of line, surface and volume elements} 
If an infinitesimal line element \(d\bm{X}\) in the reference configuration is considered along with its counterpart \(d\bm{x}\) in the current configuration, it can be shown that
\begin{equation}
    d\bm{x} = \mathbf{F} d\bm{X}, \quad
    d\bm{X} = \mathbf{F}^{-1} d\bm{x}
    \label{eq:linemap}
\end{equation}
and hence suggesting the following symbolic representation for the deformation gradient
\begin{equation}
    \mathbf{F} = \text{Grad}\,{\bm{x}} = \pdv{\bm{x}}{\bm{X}}, \quad
    \mathbf{F}^{-1} = \text{grad}\,{\bm{X}} = \pdv{\bm{X}}{\bm{x}}.
    \label{eq:defgrd_symbolic}
\end{equation}
The deformation gradient is also suitable to describe the deformation of volume and surface elements. If \(dV_{0}\) is the volume of an infinitesimal volume element in the reference configuration and \(dV\) is the corresponding volume of the same volume element after deformation in the current configuration, then it can be shown that
\begin{equation}
    \dfrac{dV}{dV_{0}} = \text{det}\mathbf{F} = |\text{F}^{i}_{.j}| =  J
    \label{eq:volumemap}
\end{equation}

In the same way if we consider vectorial surface elements \(d\bm{S}\) and \(d\bm{s}\) along with the positive unit normals to these surfaces as \(\bm{N}\) and \(\bm{s}\) in the reference and current configuration respectively, we have the relation 
\begin{equation}
    d\bm{s} = J\mathbf{F}^{-T}d\bm{S},\quad ds\bm{n} = J\mathbf{F}^{-T}\bm{N}dS
    \label{eq:nanson}
\end{equation}
which is also referred to as Nanson's formula.

\paragraph*{Strain}
A material is said to be strained in some point \(\text{P}(\bm{X})\) if at least one of the infinitesimal line elements \(d\bm{X}\) based on \(\bm{X}\) changes its length after deformation. For the square lengths we can write by virtue of \cref{eq:linemap} we have
\begin{align}
    \nonumber||d\bm{x}||^{2} 
             &= d\bm{x} \cdot d\bm{x} 
              = (\mathbf{F} d\bm{X}) \cdot (\mathbf{F} d\bm{X})\\
    \nonumber&= (d\bm{X}\mathbf{F}^{T}) \cdot (\mathbf{F} d\bm{X}) 
              = d\bm{X} (\mathbf{F}^{T} \mathbf{F}) d\bm{X}\\
             &= d\bm{X} \mathbf{C} d\bm{X},\label{eq:sqlen_cur}\\[1em]
    \nonumber||d\bm{X}||^{2} 
             &= d\bm{X} \cdot d\bm{X} 
              = (\mathbf{F}^{-1} d\bm{x}) \cdot (\mathbf{F}^{-1} d\bm{x})\\
    \nonumber&= (d\bm{x}\mathbf{F}^{-T}) \cdot (\mathbf{F}^{-1} d\bm{x}) 
              = d\bm{x} (\mathbf{F}^{-T} \mathbf{F}^{-1}) d\bm{x}\\
             &= d\bm{x} \mathbf{b}^{-1} d\bm{x},\label{eq:sqlen_ref}
\end{align}
where the symmetric tensors 
\begin{equation}
    \mathbf{C} = \mathbf{F}^{T}\mathbf{F}, \quad
    \mathbf{b} = \mathbf{F}\mathbf{F}^{T}
    \label{eq:def_right_left_cauchygreen}
\end{equation}
are referred to as the right and left Cauchy-Green deformation tensor respectively.
They can also be expressed in terms of the tangent vectors as
\begin{align}
    \mathbf{C} &= g_{ij} \;\bm{G}^{i} \otimes \bm{G}^{j}\\
    \mathbf{b} &= G^{ij} \;\bm{g}_{i} \otimes \bm{g}_{j}
\end{align}
where the abbreviations 
\begin{equation}
    g_{ij} = \bm{g}_{i} \cdot \bm{g}_{j}, \quad
    G^{ij} = \bm{G}^{i} \cdot \bm{G}^{j}, \quad
    i,j = 1,2,3
\end{equation}
In order to express strains, the difference of square lengths are in \cref{eq:sqlen_cur} and \cref{eq:sqlen_ref} are considered and after mathematical manipulation the we get the symmetric tensors
\begin{alignat}{2}
    \mathbf{E} 
    &=\dfrac{1}{2} (\mathbf{C} - \mathbf{I}) 
    &&=\dfrac{1}{2} (\mathbf{F}^{T}\mathbf{F} - \mathbf{I})\\
    \mathbf{e}
    &=\dfrac{1}{2} (\mathbf{I} - \mathbf{b}^{-1}) 
    &&=\dfrac{1}{2} (\mathbf{I} - \mathbf{F}^{-T} \mathbf{F}^{-1})
\end{alignat}

which are called the Green-Lagrange strain tensor and Almansi strain tensor respectively. The material body is thus said to be unstrained if \(\mathbf{E} = \mathbf{e} = \mathbf{0}\) and the deformation gradient represents an orthogonal tensor, and hence, \(\mathbf{F}^{T}\mathbf{F} = \mathbf{I}\). If this is the case in every point of the body, then we deal with the rigid body motion.

\paragraph*{Stretch and Shear}
With the aid of the Cauchy-Green tensor, it is possible to express the stretch of a line element as well as change in angle between two orthogonal line elements defined in the reference configuration. The ratio of the deformed to the reference length of a line element is called stretch. If \(\bm{N}\) and \(\bm{n}\) are unit vectors along the infinitesimal line element \(d\bm{X}\) and its counterpart in the current configuration \(d\bm{x}\), respectively. Furthermore with \(d\bm{X}=||d\bm{X}||\bm{N}\) and \(d\bm{x}=||d\bm{x}||\bm{n}\), the stretch in the direction of \(\bm{N}\) takes the form
\begin{align}
    \nonumber\lambda(\bm{N}) &= \sqrt{\dfrac{||d\bm{x}||^{2}}{||d\bm{X}||^{2}}}\\
                    &= {(\bm{N}\bm{C}\bm{N})}^{\textstyle{\frac{1}{2}}}
\end{align}
and similarly the strecth in the direction of \(\bm{n}\) it can be shown
\begin{equation}
    \lambda(\bm{n}) = {(\bm{n}\bm{b}^{-1}\bm{n})}^{\textstyle-{\frac{1}{2}}}
\end{equation}
By representing the Cauchy-Green tensor with respect to an orthonormal basis \(\bm{N}_{i} \cdot \bm{N}_{j} = \delta_{ij},\;(i,j = 1,2,3)\), we get 
\begin{equation}
    \lambda(\bm{N}_i) 
    = {(\bm{N}_i\bm{C}\bm{N}_i)}^{\textstyle{\frac{1}{2}}}
    = \sqrt{\text{C}_{ii}}, \quad \text{no sum over } i = 1,2,3
\end{equation}
Thus the square roots of the diagonal components of the Cauchy-Green tensor expressed in an orthonormal basis represent the stretches in the corresponding directions \(\bm{N}_i\;(i=1,2,3)\)
The decrease in the angle after deformation between two orthogonal line elements, say \(d\bm{X}_{i}=||d\bm{X}_{i}||\bm{N}_{i},\: (i=1,2)\) defined in the reference configuration is given by
\begin{equation}
    \sin\varphi_{12} = \dfrac{\text{C}_{12}}{\sqrt{\text{C}_{11}\text{C}_{22}}}
\end{equation}

\paragraph*{Spectral Decomposition of Strain Tensors}
By setting up an eigen value problem for the right Cauchy-Green tensor we have
\begin{equation}
    \mathbf{C} \bm{N} = \Lambda \bm{N}, \quad \bm{N} \neq 0
\end{equation}
where a non-zero vector \(\bm{N}\) is called an eigen vector and \(\Lambda\) denotes the corresponding eigen value. To solve this eigen value problem, \(\mathbf{C}\) and \(\bm{N}\) are represented with respect to a basis. This finally results in 
\begin{equation}
    (\text{C}^{i}_{j} - \Lambda\delta^{i}_{j})N^{j} = 0, \quad
    i = 1,2,3
\end{equation}
which is a homogenous linear equation system with respect to the components of the eigen vector \(\bm{N}\). This equation system has a non-trivial solution if an only if
\begin{equation}
    |\text{C}^{i}_{j} - \Lambda\delta^{i}_{j}| = 0
\end{equation}
or equivalently
\begin{equation}
    \begin{vmatrix}
        \text{C}^{1}_{1}  - \Lambda & \text{C}^{1}_{2} & \text{C}^{1}_{3} \\[0.5em]
        \text{C}^{2}_{1} & \text{C}^{2}_{2} - \Lambda & \text{C}^{2}_{3} \\[0.5em]
        \text{C}^{3}_{1} & \text{C}^{3}_{2} & \text{C}^{3}_{3} - \Lambda
    \end{vmatrix} = 0,
\end{equation}
where \(|\bullet|\) denotes the determinant of a matrix. Writing out this determinant and collecting the terms according to the powers of \(\Lambda\) one obtains the so-called characteristic equation which is given by
\begin{equation}
    \Lambda^{3} 
    - \text{I}_{\mathbf{C}}\Lambda^{2}
    + \text{II}_{\mathbf{C}}\Lambda
    - \text{III}_{\mathbf{C}} = 0,
\end{equation}
where the coefficients take the form
\begin{align}
    \text{I}_{\mathbf{C}} &= \text{C}_{i}^{i} = \text{tr}\mathbf{C} \\
    \text{II}_{\mathbf{C}} &= \dfrac{1}{2}(\text{C}^{i}_{i}\text{C}^{j}_{j} - \text{C}^{i}_{j}\text{C}^{j}_{i}) = \dfrac{1}{2}[(\text{tr}\mathbf{C})^{2} - \text{tr}\mathbf{C}^{2}]\\  
    \text{III}_{\mathbf{C}} &= |\text{C}^{i}_{j}| = \text{det}\mathbf{C}
\end{align}
The solution of the eigen value problem resulting from the characteristic equation does not depend on the choice of the coordinate system. Therefore, also the coefficients of the characteristic equation are independent of the coordinate system and, hence, are called the principal invariants of \(\mathbf{C}\). Since the tensor \(\mathbf{C}\) is symmetric its characteristic equation has three real roots \(\Lambda_{i}(i=1,2,3)\). The characteristic equation can also be represented in terms of the roots by 
\begin{equation}
    (\Lambda_{1} - \Lambda)(\Lambda_{2} - \Lambda)(\Lambda_{3} - \Lambda) = 0
\end{equation}
By virtue of the tensor identity, \(\text{tr}(\mathbf{AB}) = \text{tr}(\mathbf{BA})\) and taking \cref{eq:volumemap} into account it can be shown that
\begin{align}
    \text{tr}\mathbf{C} &= \text{tr}\mathbf{b}\\
    \text{tr}\mathbf{C}^2 &= \text{tr}\mathbf{b}^2\\ 
    \text{det}\mathbf{C} &= \text{det}\mathbf{b} = J^2
\end{align}
Thus, we conclude the tensors \(\mathbf{C}\) and \(\mathbf{b}\) have the same principal invariants
\begin{equation}
    \text{I}_{\mathbf{C}} = \text{I}_{\mathbf{b}}, \quad 
    \text{II}_{\mathbf{C}} = \text{II}_{\mathbf{b}}, \quad
    \text{III}_{\mathbf{C}} = \text{III}_{\mathbf{b}}
    \label{eq:straininv}
\end{equation}
which are also referred to as strain invariants. \cref{eq:straininv} immediately implies the equivalence of the eigen values of the right and left Cauchy-Green tensors. This is an important property and is useful for material modelling.

The eigen vectors of symmetric tensors such as \(\mathbf{C}\) and \(\mathbf{b}\) are linearly independent and can be represented by an orthonormal basis. Thus, we can write the spectral decomposition as
\begin{equation}
    \mathbf{C} = \sum_{n = 1}^{3}\Lambda_{i}\bm{N}_{i} \otimes \bm{N}_{i}, \quad
    \mathbf{b} = \sum_{n = 1}^{3}\Lambda_{i}\bm{n}_{i} \otimes \bm{n}_{i},
    \label{eq:spectraldecomposition}
\end{equation}
where
\begin{equation}
    \bm{N}_{i} \cdot \bm{N}_{j} = \delta_{ij}, \quad
    \bm{n}_{i} \cdot \bm{n}_{j} = \delta_{ij}, \quad
    i,j = 1,2,3.
\end{equation}
In view of \cref{eq:sqlen_cur} and \cref{eq:sqlen_ref} the tensors \(\mathbf{C}\) and \(\mathbf{b}\) are positive definite which implies that their eigen values \(\Lambda_{i}(i=1,2,3)\) are positive. Using this property we can write the powers of these tensors on the basis of the spectral decomposition by
\begin{equation}
    \mathbf{C}^{\alpha} = \sum_{n = 1}^{3}\Lambda_{i}^{\alpha}\bm{N}_{i} \otimes \bm{N}_{i}, \quad
    \mathbf{b}^{\alpha} = \sum_{n = 1}^{3}\Lambda_{i}^{\alpha}\bm{n}_{i} \otimes \bm{n}_{i},
    \label{eq:power_C}
\end{equation}
\begin{figure}[htpb]
    \centering
    \includegraphics[width=\textwidth]{polar_decompositioin.pdf}
    \caption{Polar decomposition}%
    \label{fig:polar_decompositioin}
\end{figure}

\paragraph*{Polar Decomposition of Deformation Gradient} 
The deformation gradient can be further decomposed into a rotation and stretch tensor. By setting in \cref{eq:power_C} \(\alpha=\frac{1}{2}\) one defines the so-called right and left stretch tensors
\begin{equation}
    \mathbf{U} 
    = \mathbf{C}^{\frac{1}{2}} 
    = \sum_{i=1}^{3}  \lambda_{i} \bm{N}_{i} \otimes \bm{N}_{i}, \quad
    \mathbf{v} 
    = \mathbf{b}^{\frac{1}{2}} 
    = \sum_{i=1}^{3}  \lambda_{i} \bm{n}_{i} \otimes \bm{n}_{i},
    \label{eq:defUv}
\end{equation}
where 
\begin{equation}
    \lambda_{i} = \sqrt{\Lambda_{i}}, \quad i=1,2,3.
\end{equation}
Further one can show that 
\begin{equation}
    \mathbf{R} = \mathbf{F}\mathbf{U}^{-1}
    \label{eq:rotation}
\end{equation}
represents a proper orthogonal tensor as it can be shown that \(\mathbf{R}\mathbf{R}^{T}=\mathbf{I}\) and \(\text{det}(\mathbf{R}) \geq 0\). 
From \cref{eq:rotation} we further get
\begin{equation}
    \mathbf{F} 
    = \mathbf{R}\mathbf{U} 
    = (\mathbf{R}\mathbf{U}\mathbf{R}^{T})\mathbf{R}.
\end{equation}
Due to the symmetry of \(\mathbf{U}\) it can be shown that the tensor \(\mathbf{R}\mathbf{U}\mathbf{R}^{T}\) is also symmetric and one can derive that \((\mathbf{R}\mathbf{U}\mathbf{R}^{T})^{2} = \mathbf{b}\). In view of \cref{eq:defUv}{\({}_2\)}, there exists only one real symmetric tensor whose square is \(\mathbf{b}\). Hence \(\mathbf{R}\mathbf{U}\mathbf{R}^{T} = \mathbf{v}\). This results in the following identity
\begin{equation}
    \mathbf{F} 
    = \mathbf{R}\mathbf{U}
    = \mathbf{v}\mathbf{R}
    \label{eq:polar_decomposition}
\end{equation}
which is referred to as the \emph{polar decomposition} of the deformation gradient. By virtue of \cref{eq:polar_decomposition} we also obtain
\begin{equation}
    \mathbf{U} = \mathbf{R}^{T}\mathbf{v}\mathbf{R}, \quad
    \mathbf{C} = \mathbf{R}^{T}\mathbf{b}\mathbf{R}, \quad
    \mathbf{v} = \mathbf{R}\mathbf{U}\mathbf{R}^{T}, \quad
    \mathbf{b} = \mathbf{R}\mathbf{C}\mathbf{R}^{T}
    \label{eq:rotation_stretch_cauchy}
\end{equation}
Considering a special case of pairwise distinct eigen values of the stretch tensors \(\lambda{1}\neq \lambda{2} \neq \lambda{3}\neq \lambda{1}\) and using \cref{eq:rotation_stretch_cauchy} it can be shown that 
\begin{equation}
    \mathbf{v} = \mathbf{R}\mathbf{U}\mathbf{R}^{T} 
    = \sum_{i=1}^{3}  \lambda_{i} (\mathbf{R}\bm{N}_{i}) \otimes (\mathbf{R}\bm{N}_{i})
\end{equation}
Comparing this with \cref{eq:defUv}\({}_2\) we obtain
\begin{equation}
    \bm{n}_{i} = \mathbf{R}\bm{N}_{i}, \quad
    \bm{V}_{i} = \mathbf{R}^{T}\bm{n}_{i}, \quad
    i = 1,2,3.
\end{equation}
and consequently
\begin{equation}
    \mathbf{R} = \sum_{i=1}^{3}  \bm{n}_{i} \otimes \bm{N}_{i}, \quad
    \mathbf{F} = \sum_{i=1}^{3}  \lambda_{i} \bm{n}_{i} \otimes \bm{N}_{i}, \quad
    \prstretche{}{1} \neq \prstretche{}{2} \neq \prstretche{}{3} \neq \prstretche{}{1} 
\end{equation}
Thus the deformation of an eigen vector \(\bm{N}_{i} (i=1,2,3)\) can be accomplished in two steps: rotation by \(\mathbf{R}\) to \(\bm{n}_{i}\) and stretching by \(\mathbf{v}\) to \(\lambda\bm{n}_{i}\) or vice versa: stretching by \(\mathbf{U}\) to \(\lambda\bm{N}_{i}\) and rotation by \(\mathbf{R}\) to \(\lambda\bm{n}_{i}\) as shown illustratively in \cref{fig:polar_decompositioin} and mathematically as
\begin{alignat}{3}
    \nonumber\mathbf{F}\bm{N}_{i} 
    &= \mathbf{v}(\mathbf{R}\bm{N}_{i}) 
    &&=  \mathbf{v}\bm{\lambda}_{i} 
    &&= \lambda_{i}\bm{n}_{i} \quad \text{or} \\
    \mathbf{F}\bm{N}_{i}
    &= \mathbf{R}(\mathbf{U}\bm{N}_{i}) 
    &&= \lambda_{i}\mathbf{R}\bm{N}_{i}
    &&= \lambda_{i}\bm{n}_{i}, \quad
    i = 1,2,3
\end{alignat}

For this reason, \(\mathbf{R}\) is called the rotation tensor and the eigen values \(\lambda_{i}(i=1,2,3)\) of the stretch tensors \(\mathbf{U}\) and \(\mathbf{v}\) are referred to as the principal stretches.
\paragraph*{Velocity Gradient}
The velocity of a material particle P is defined by
\begin{equation}
    \bm{v} = \pdv{\bm{x}(\bm{X},t)}{t} = \dot{\bm{x}}
    \label{eq:velocity}
\end{equation}
The time derivative denoted by the superposed dot is called the material time derivative since the position vector \(\mathbf{X}\) in the reference configuration is held constant. The material velocity gradient is defined similar to the deformation gradient by
\begin{equation}
    \mathbf{L} = \text{Grad}\bm{v} = \pdv{\bm{v}}{\bm{X}}.\
    \label{eq:def_matVelGradient}
\end{equation}
Taking into account \cref{eq:velocity} and \cref{eq:defgrd_symbolic} we can write 
\begin{equation}
    \mathbf{L} = \text{Grad}\dot{\bm{x}} 
    = \pdv*{\left[\pdv{\bm{x}(\bm{X},t)}{\bm{t}}\right]}{\bm{X}}
    = \pdv*{\left(\pdv{\bm{x}}{\bm{X}}\right)}{t}
    = \dot{\mathbf{F}}
    \label{eq:matVelGradient}
\end{equation}
Of special interest is also the spatial velocity gradient defined by 
\begin{equation}
    \mathbf{l} = \text{grad}\bm{v} = \pdv{\bm{v}}{\bm{x}}
    \label{eq:def_spatVelGradient}
\end{equation}
Using \cref{eq:defgrd_symbolic}\({}_{2}\) and \cref{eq:def_matVelGradient} we thus obtain
\begin{equation}
    \mathbf{l} = \pdv{\bm{v}}{\bm{X}}\pdv{\bm{X}}{\bm{x}}
    = \dot{\mathbf{F}}\mathbf{F}^{-1}
    \label{eq:spatVelGradient}
\end{equation}
According to \cref{eq:def_matVelGradient} and \cref{eq:def_spatVelGradient} it follows that 
\begin{equation}
    d\bm{v} = d\dot{\bm{x}} = \mathbf{l}d\bm{x} = \mathbf{L}d\bm{X}.
\end{equation} 
Furthermore, the spatial gradient of velocity is usually decomposed into a symmetric \(\mathbf{d}=\mathbf{d}^{T}\) and a skew-symmetric part \(\mathbf{w}=-\mathbf{w}^{T}\) by 
\begin{equation}
    \mathbf{l} = \mathbf{d} + \mathbf{w},
    \label{spatial_velocity_gradient}
\end{equation} 
where
\begin{align}
    \nonumber\mathbf{d} 
    &= \dfrac{1}{2}\left(\mathbf{l}+\mathbf{l}^{T}\right) 
    = \dfrac{1}{2}\left(\dot{\mathbf{F}}\mathbf{F}^{-1} +\mathbf{F}^{-T}\dot{\mathbf{F}}^{T}\right)
    = \dfrac{1}{2}\mathbf{F}^{-T}\dot{\mathbf{C}}\mathbf{F}^{-1}\\
    &= \mathbf{F}^{-T}\dot{\mathbf{E}}\mathbf{F}^{-1}
    \label{eq:rate_of_deformation}
\end{align}
is called the rate of deformation tensor and 
\begin{equation}
    \nonumber\mathbf{w} 
    = -\mathbf{w}^{T}
    = \dfrac{1}{2}\left(\mathbf{l}-\mathbf{l}^{T}\right)
    = \dfrac{1}{2}\left(\dot{\mathbf{F}}\mathbf{F}^{-1} -\mathbf{F}^{-T}\dot{\mathbf{F}}^{T}\right)
    \label{vorticity}
\end{equation}
is called spin (vorticity) tensor.

\subsection{Stress}
\begin{figure}[htpb]
    \centering
    \includegraphics[width=0.7\textwidth]{cauchy_stress_vector.pdf}
    \caption{Cauchy stress vector}%
    \label{fig:cauchy_stress_vector}
\end{figure}
Let us consider a body B in the current configuration at a time \(t\). In order to define stress in some point P let us further imagine a smooth surface going through P and separating B into two parts, of which one part is as shown in \cref{fig:cauchy_stress_vector}. One can then define a force \(\Delta\bm{p}\) and a couple \(\Delta\bm{m}\) resulting from the forces exerted by material on one side of the surface \(\Delta A\) on the material on the other side. Let the area \(\Delta A\) tend to zero permanently keeping P as an inner point. A basic postulate of continuum mechanics is that the limit 
\begin{equation}
    \bm{t} = \lim_{\Delta A \to 0}\frac{\Delta\bm{p}}{\Delta A}
\end{equation}
exists and is finite. The so-defined vector \(\bm{t}\) is called the Cauchy stress vector. The Cauchy's fundamental postulate states that the vector t depends on the surface only through the unit outward normal \(\bm{n}\) as shown in \cref{fig:cauchy_stress_vector}. In other words, the Cauchy stress vector is the same for all surface through P which have a normal \(\bm{n}\) in P. As a result one can write
\begin{equation}
    \bm{t} = \bm{t}(\bm{x},\bm{n}) = \bm{t}(\bm{X},\bm{n},t).
    \label{eq:cauchy_stress_vector}
\end{equation}
According to Cauchy's theorem the mapping \(\bm{n}\rightarrow \bm{t}\) is linear provided the function \cref{eq:cauchy_stress_vector} is continuous at \(\bm{X}\). Hence this mapping can be described by a second-order tensor as 
\begin{equation}
    \bm{t} = \bm{\sigma}\bm{n}
    \label{eq:cauchy_stress_tensor}
\end{equation}
The tensor \(\bm{\sigma}\) is called the Cauchy stress tensor and is related to the surface in the current configuration. 

\begin{figure}[htpb]
    \centering
    \includegraphics[width=\textwidth]{first_piola_kirchoff.pdf}
    \caption{First Piola-Kirchoff stress tensor}%
    \label{fig:first_piola_kirchoff_stress}
\end{figure}
Sometimes it is useful to define a stress tensor with respect to the reference configuration. For this consider a function \(d\bm{p}=d\bm{p}(d\bm{A})\) where
\begin{equation}
    d\bm{p} = \bm{t}dA
\end{equation}
and \(d\bm{A}=\bm{n}dA\) represents a vectorial surface element corresponding to the differential area \(dA\). The function \(\bm{t} = \bm{t}(\bm{n})\) is linear if and only if the function \(d\bm{p}=d\bm{p}(d\bm{A})\) is linear. Furthermore, using Nanson's formula we have
\begin{equation}
    d\bm{p} = \bm{t}dA = \bm{\sigma}d\bm{A} = J\bm{\sigma}\mathbf{F}^{-T}d\bm{A}_{0}
\end{equation}
With the aid of the definition
\begin{equation}
    \mathbf{P}=J\bm{\sigma}\mathbf{F}^{-T} = \bm{\tau}\mathbf{F}^{-T}
    \label{eq:def_firstpiolakirchoffstress}
\end{equation}
where 
\begin{equation}
    \bm{\tau} = J\bm{\sigma}
    \label{eq:kirchoff_stress}
\end{equation}
is referred to as the Kirchoff stress tensor, we can write
\begin{equation}
    d\bm{p} = \mathbf{P}d\bm{A}_{0}.
    \label{eq:firstpiolakirchoffmap}
\end{equation}
\(\mathbf{P}\) is called the first Piola-Kirchoff stress tensor and it relates the stress vector to the surface in the reference configuration as shown in \cref{eq:firstpiolakirchoffmap}. 
Furthermore the second Piola-Kirchoff stress tensor is defined by 
\begin{equation}
    \mathbf{S}  = J\mathbf{F}^{-1}\bm{\sigma}\mathbf{F}^{-T}
    =\mathbf{F}^{-1} \tautot \mathbf{F}^{-T}
     = \mathbf{F}^{-1}\mathbf{P}
    \label{eq:secondpiolakirchoff_stress}
\end{equation}
which is an important stress tensor quantity but has no physical significance.

\subsection{Balance Laws}
\paragraph*{Linear Momentum Balance}
If we consider the current configuration of the body B with the volume \(V\), mass \(M\) and boundary surface \(A\), the linear momentum of the body is defined by 
\begin{equation}
    \int_{M}^{} \bm{v} \,dm 
\end{equation}
where \(\bm{v} = \pdv{\bm{x}}{t} = \dot{\bm{x}}\) denotes the velocity of a particle with the position vector \(\bm{x}\) and mass \(dm\). Using the relation \(dm = \rho dV\), where \(\rho\) denotes the density, we can write 
\begin{equation}
    \int_{M}^{} \bm{v} \,dm 
    =  \int_{M}^{} \dot{\bm{x}} \,dm 
    =  \int_{V}^{} \rho\dot{\bm{x}} \,dV 
\end{equation}
According to a fundamental principle of continuum mechanics the rate of change of the linear momentum is equal to the resultant force applied on the body. This resultant force comprises a body force (without inertia) and a surface force and is thus written by 
\begin{equation}
    \int_{V}^{} \bm{f} \,dV 
    + \int_{A}^{} \bm{t} \,dA
\end{equation}
Thus, the linear momentum balance (the first Euler law of motion) takes the form
\begin{equation}
    \pdv*{\int_{V}^{} \rho\dot{\bm{x}} \,dV }{t}
    = \int_{V}^{} \bm{f} \,dV 
    + \int_{A}^{} \bm{t} \,dA
    \label{eq:lin_momentum_balance}
\end{equation}
Using the mass conservation law we can rewrite the left hand side of \cref{eq:lin_momentum_balance} by
\begin{equation}
    \pdv*{\int_{V}^{} \rho\dot{\bm{x}} \,dV }{t} 
    = \pdv*{ \int_{M}^{} \bm{v} \,dm}{t} 
    =  \int_{M}^{} \dot{\bm{v}} \,dm 
    = \int_{V}^{} \rho\ddot{\bm{x}} \,dV
    = \int_{V}^{} \rho \bm{a} \,dV
\end{equation}
where \(\bm{a} = \dot{\bm{v}} = \ddot{\bm{a}}\) represents the acceleration vector of particle, which finally leads to 
\begin{equation}
    \int_{V}^{} \rho\ddot{\bm{x}} \,dV = \int_{V}^{} \bm{f} \,dV 
    + \int_{A}^{} \bm{t} \,dA
\end{equation}
By means of the Cauchy theorem \cref{eq:cauchy_stress_tensor} and Gauss divergence theorem from calculus we can write 
\begin{equation}
        \int_{V}^{} \rho\ddot{\bm{x}} \,dV
        = \int_{V}^{} \bm{f} \,dV 
        + \int_{V}^{} \text{div}\bm{\sigma} \,dV
\end{equation}
and consequently
\begin{equation}
    \int_{V}^{} (\text{div}\bm{\sigma} + \bm{f} -\rho\ddot{\bm{x}}) \,dV
    \label{eq:simplified_lin_momentum_balance}
\end{equation}
This equation holds for the whole body and any arbitrary part of it with the volume \(dV\). Assuming that the integrand in \cref{eq:simplified_lin_momentum_balance} is a continuous function in space we have
\begin{equation}
    \text{div}\bm{\sigma} + \bm{f} = \rho\ddot{\bm{x}}
    \label{eq:result_lin_momentum_balance}
\end{equation}
The equation of motion \cref{eq:result_lin_momentum_balance} can also be written in terms of the first Piola-Kirchoff stress tensor \cref{eq:def_firstpiolakirchoffstress} and is given by 
\begin{equation}
    \text{Div}\mathbf{P} + \bm{f}_{0} = \rho_{0}\ddot{\bm{x}}
    \label{eq:result_lin_momentum_balance_2}
\end{equation}

\paragraph*{Rotational Momentum Balance}
The rotational momentum (or moment of momentum) is defined by 
\begin{equation}
    \int_{M}^{}\bm{x} \times \bm{v} \,dm 
    =  \int_{M}^{}\bm{x} \times \dot{\bm{x}} \,dm 
    =  \int_{V}^{}\bm{x} \times \rho\dot{\bm{x}} \,dV 
\end{equation}
where \(\bm{x}\) denotes a position vector with respect to arbitrary origin O. The balance law of rotational momentum is another fundamental principle of continuum mechanics stating that the rate of the rotational momentum
\begin{equation}
    \pdv*{\int_{M}^{} \bm{x} \times \dot{\bm{x}} \,dm }{t} 
    = \int_{M}^{} \dot{\bm{x}} \times \dot{\bm{x}} \,dm 
    + \int_{M}^{} \bm{x} \times \ddot{\bm{x}} \,dm 
    = \int_{M}^{} \rho \bm{x} \times \ddot{\bm{x}} \,dV
\end{equation}
is equal to the resultant of the external forces with respect to the same origin 
\begin{equation}
    \int_{A}^{}\bm{x} \times \bm{t} \,dA
    + \int_{V}^{}\bm{x} \times \bm{f} \,dV 
\end{equation}
Thus the rotational momentum balance can be written by 
\begin{equation}
    \int_{M}^{} \rho \bm{x} \times \ddot{\bm{x}} \,dV  
    = \int_{A}^{}\bm{x} \times \bm{t} \,dA
    + \int_{V}^{}\bm{x} \times \bm{f} \,dV 
\end{equation}
A consequence of this relation is (without proof) the symmetry of the Cauchy stress tensor
\begin{equation}
    \bm{\sigma} = \bm{\sigma}^{T}
\end{equation}
\paragraph*{Balance of Mechanical Energy}
The balance of mechanical energy can be formulated from the Cauchy equation of motion given by \cref{eq:result_lin_momentum_balance}. Multiplying this equation scalarly with the velocity vector \(\bm{v} = \dot{\bm{x}}\) we get
\begin{equation}
    \bm{v} \cdot \text{div}\bm{\sigma} + \bm{v} \cdot \bm{f} 
    = \rho\bm{v} \cdot \ddot{\bm{x}}
\end{equation}
Transforming this equation using tensor calculus identities and the symmetry property of the Cauchy stress tensor, we get
\begin{equation}
    \text{div}(\bm{v\sigma}) - \bm{\sigma} : \mathbf{d} 
    + \bm{v} \cdot \bm{f} 
    = \rho\odv*{\left(\frac{1}{2}\bm{v} \cdot \bm{v}\right)}{t}
\end{equation}
Further, integration over the volume \(V\) of the body and applying the Gauss divergence theorem of calculus leads to the equation
\begin{equation}
    \odv*{\int_{M}^{}\left(\frac{1}{2}\bm{v} \cdot \bm{v}\right) \, dm}{t}
     + \int_{V}^{}\left(\bm{\sigma} : \mathbf{d}\right) \, dV
     = \int_{A}^{}\left(\bm{v} \cdot \bm{t}\right) \, dA
     + \int_{V}^{}\left(\bm{v} \cdot \bm{f}\right) \, dV
    \label{eq:result_energy_balance}
\end{equation}
expressing the balance of mechanical energy. Every term in \cref{eq:result_energy_balance} has a special physical meaning connected with the mechanical energy:
\begin{align}
    K &= \int_{M}^{}\left(\frac{1}{2}\bm{v} \cdot \bm{v}\right) \, dm
    && - \text{kinetic energy,} \\
    W &= \int_{V}^{}\left(\bm{\sigma} : \mathbf{d}\right) \, dV 
    && - \text{stress power,} \label{eq:stress_power}\\
    P &= \int_{A}^{}\left(\bm{v} \cdot \bm{t}\right) \, dA
    + \int_{V}^{}\left(\bm{v} \cdot \bm{f}\right) \, dV
    && - \text{power of external forces.} 
\end{align}
With the above abbreviations the balance of mechanical energy \cref{eq:result_energy_balance} can be given in a short form as
\begin{equation}
    \dot{K} + W = P,
\end{equation}
indicating that the power of external forces goes into the stress power (also referred to as power of internal forces) and the change of the kinetic energy.
The stress power can further be decomposed by 
\begin{equation}
    W = \mathcal{D} + \int_{V_{0}}^{} \dot{\Psi } \, dV_{0}
    \label{eq:stress_power_dissipation}
\end{equation}
where \(\mathcal{D}\) denotes the heat production (dissipation) and \(\Psi\) is the stored energy density related to the unit volume in the reference configuration.

\paragraph*{Work-conjugate Stress-Strain pairs}
From \cref{eq:result_energy_balance} it is seen that the stress power is expressed by the term \(\bm{\sigma} : \mathbf{d}\), where the rate of deformation tensor \(\mathbf{d}\) \cref{eq:rate_of_deformation} is not a material time derivative of a strain. Instead of \cref{eq:stress_power}, the stress power can be given in terms of a stress \(\mathbf{Y}\) and the material time derivative of a strain \(\mathbf{Z}\) by 
\begin{equation}
    W = \int_{V_{0}}^{}\left(\mathbf{Y} : \dot{\mathbf{Z}}\right) \, dV_{0}
    = \int_{V}^{}J^{-1}\left(\mathbf{Y} : \dot{\mathbf{Z}}\right) \, dV
\end{equation}
The pair of such stress and strain variables is called work-conjugate. It can be shown that 
\begin{equation}
    \bm{\sigma} : \mathbf{d} 
    = J^{-1} \mathbf{S} : \frac{1}{2} \dot{\mathbf{C}}
    = J^{-1} \mathbf{S} : \dot{\mathbf{E}}
    = J^{-1} \mathbf{P} : \dot{\mathbf{F}}
    \label{eq:work_conjugates}
\end{equation}
where \((\mathbf{S} , \mathbf{E})\), \((\mathbf{S} ,\frac{1}{2} \mathbf{C})\) and \((\mathbf{P} , \mathbf{F})\) are work-conjugate.

\subsection{Hyperelastic Materials}
Considering again the stress power defined by \cref{eq:stress_power} and keeping \cref{eq:work_conjugates} in mind we have
\begin{equation}
    W = \int_{V}^{}\left(\bm{\sigma} : \mathbf{d}\right) \, dV 
      = \int_{V_0}^{}\left(\mathbf{S} : \dot{\mathbf{E}}\right) \, dV_{0}
      = \int_{V_0}^{}\left(\mathbf{P} : \dot{\mathbf{F}}\right) \, dV_{0} 
      \label{eq:stress_power_alternate}
\end{equation}
Generally it is not integrable over time, i.e. there is no scalar function \(\Psi\) such that 
\begin{equation}
    W = \int_{V_0}^{} \dot{\Psi} \, dV_{0}
    \label{eq:def_hyperelastic}
\end{equation}
A material for which such a function \(\Psi = \hat{\Psi}(\mathbf{F})\) exists is called \emph{hyperelastic} or Green elastic. The function \(\Psi\) is then referred to as elastic potential or strain energy function. It is further seen from \cref{eq:stress_power_dissipation}
\begin{equation}
    \mathcal{D} = 0
\end{equation}
The elastic potential has to satisfy the material objectivity condition and, hence, it can be shown that
\begin{equation}
    \Psi = \hat{\Psi}(\mathbf{U}) = \hat{\Psi}(\mathbf{C})
\end{equation}
Comparing \cref{eq:def_hyperelastic} with \cref{eq:stress_power_alternate} and using the chain rule of differentiation one obtains the relations
\begin{equation}
    \mathbf{P} = \pdv{\Psi}{\mathbf{F}}, \quad
    \mathbf{S} = \pdv{\Psi}{\mathbf{E}}
               = 2 \pdv{\Psi}{\mathbf{C}}
    \label{eq:stress_from_strain_energy}
\end{equation}

It can be shown that an isotropic tensor function \(\Psi\) has the same value for all symmetric tensors with the same eigen values. Thus, this function can uniquely be defined by the eigen values of the tensor argument. The same valid for the principal invariants and principal traces because they can be expressed in terms of the eigen values in the unique form. This implies that for isotropic hyperelastic materials, the tensor function \(\Psi\) can be represented in terms of the strain invariants \(\text{I}_{\mathbf{C}}, \text{II}_{\mathbf{C}}, \text{III}_{\mathbf{C}}\), principal stretches \(\lambda_{i}\) or principal traces tr\(\mathbf{C}^{i} (i=1,2,3)\) by
\begin{equation}
    \Psi = \Psi(\text{I}_{\mathbf{C}}, \text{II}_{\mathbf{C}}, \text{III}_{\mathbf{C}})
         = \Psi(\lambda_{1}, \lambda_{2}, \lambda_{3})
         = \Psi(\text{tr}\mathbf{C}^{1}, \text{tr}\mathbf{C}^{2}, \text{tr}\mathbf{C}^{3})
\end{equation}

By using different functions \(\Psi\), various isotropic hyperelastic strain energy functions have been proposed. One such class of functions is given by the \emph{Ogden} model 
\begin{equation}
    \Psi = \sum_{r = 1}^{s} \frac{\mu_{r}}{\alpha_{r}} (\lambda_{1}^{\alpha_{r}} + \lambda_{2}^{\alpha_{r}} + \lambda_{3}^{\alpha_{r}} - 3), 
\end{equation}
where \(\mu_{r}\) and \(\alpha_{r} (r =1,\ldots, s)\) represent material constants.

\section{Finite Viscoelasticity Theory}
\label{sec:finite_viscoelasticity}
In this section a summary of the most important aspects of the Finite Viscoelasticity theory by Reese and Govindjee \cite{Reese1998Sep} is presented. This theory is valid for deviations of any size from the thermodynamic equilibrium, i.e. the strain rates do not have to be close to zero. Furthermore it involves a \emph{multiplicative split} of the deformation gradient \(\mathbf{F}\) and an \emph{additive split} of the strain energy function \(\Psi\). The evolution equation being similar to Finite Elastoplasticity is integrated using an exponential mapping algorithm, which is linear in the logarithmic principle values of a kinematic measure \(\mathbf{b}\). It also leads to a symmetric tangent stiffness modulus due to derivation of the viscous stress (or over-stress) from a potential and usage of an evolution equation that has a special form.

\subsection{Multiplicative Split of Deformation Gradient}
In order to account for both elastic and viscous material behaviour at large deformations, the deformation gradient is split multiplicatively into an elastic and inelastic (or viscous in the case of viscoelasticity) part. For the case of several relaxation mechanisms \((k=1,2,\ldots,N)\) it is given by
\begin{equation} 
    \mathbf{F}={\Fe}^{k} {\Fi}^{k}
    \label{eq:dfgrd_mul_split}
\end{equation}
Using \cref{eq:dfgrd_mul_split} it follows that
\begin{align}
    \nonumber\Ctot
    &=\tF \Ftot ={({\Fi}^{k} {\Fe}^{k})}^{T} {\Fe}^{k} {\Fi}^{k}\\
    &={{\Fi}^{k}}^{T} ({{\Fe}^{k}}^{T} {\Fe}^{k}) {\Fi}^{k}
    ={{\Fi}^{k}}^{T} {\Ce}^{k} {\Fi}^{k}, \quad k =1,2,\ldots,N
\end{align}
and thus 
\begin{equation}
    {\Ce}^{k}
     = {{\Fe}^{k}}^{T} {\Fe}^{k}
     = {{\Fi}^{k}}^{-T} {\Ctot} {{\Fi}^{k}}^{-1}, \quad
    k =1,2,\ldots,N.
    \label{eq:Ce_k}
\end{equation}
Furthermore the strain energy function is split additively into an equilibrium part (elastic) and a non-equilibrium part (viscous) given by
\begin{align}
    \nonumber\Psi 
    &= \Psieq(\Ctot) + \sum_{k=1}^{N}{\Psineq}^{k}({\Ce}^{k}) \\
    \nonumber
    &= \Psieq(\Ctot) + \sum_{k=1}^{N}{\Psineq}^{k}({{\Fi}^{k}}^{-T} {\Ctot}^{k} {{\Fi}^{k}}^{-1}) \\
    &= \Psi(\Ctot, {\Fi}^{1}, {\Fi}^{2}, \ldots {\Fi}^{N}) 
    \label{eq:strain_energy_function_tot}
\end{align}
which is a function of the right Cauchy-Green deformation tensor \cref{eq:def_right_left_cauchygreen} and a set of \(N\) internal variables \({\Fi}^{k} (k = 1,2,\ldots,N)\). In order to determine the internal variables, a set of \(N\) evolution equations of the form 
\begin{equation}
    {\dot{\Fi}}^{k} = {f}^{k}(\Ctot, {\Fi}^{1}, {\Fi}^{2}, \ldots {\Fi}^{N}), \quad
    k =1,2,\ldots,N.
\end{equation}

\subsection{Derivation of Evolution Equation}
For the sake of ease the simple case of only a single relaxation mechanism \((k=1)\) will be considered in deriving the further equations, wherin
\begin{equation}
    \Ftot={\Fe} {\Fi}, \quad
    \Ce = {{\Fi}}^{-T} {\Ctot} {{\Fi}}^{-1}, \quad
    \Psi = \Psi(\Ctot, {\Fi}), \quad
    \dot{\Fi} = f(\Ctot, {\Fi}) 
    \label{eq:visc_simple_case}
\end{equation}
However, these results can be extended to incorporate several relaxation mechanisms \((k \geq 1)\).

According to the second law of thermodynamics in every real process 
\begin{equation}
    \mathcal{D} \geq 0
\end{equation}
where \(\mathcal{D}\) denotes the dissipation rate. By means of \cref{eq:stress_power_dissipation} and \cref{eq:work_conjugates} this inequality can be expressed by 
\begin{equation}
    \Stot : \frac{1}{2} \dot\Ctot - \dot{\Psitot} \geq 0
    \label{eq:dissipation inequality}
\end{equation}
where 
\begin{equation}
    \Stot = \Stot(\Ctot, {\Fi})
\end{equation} is the second Piola-Kirchoff stress tensor. By using \cref{eq:visc_simple_case}, \cref{eq:dissipation inequality} and identities for derivatives of tensor functions \cite[refer Eq. 1.149][]{Itskov} we have
\begin{equation}
     \Stot : \frac{1}{2} \dot{\Ctot} 
    - \pdv{\Psieq}{\Ctot} : \dot{\Ctot}
    - \pdv{\Psineq}{\Ce} : \pdv{\Ce}{\Fi} : \dot{\Fi}
    - \pdv{\Psineq}{\Ce} : \pdv{\Ce}{\Ctot} : \dot{\Ctot} \geq 0
\end{equation}
where
\begin{equation}
    \pdv{\Ce}{\Ctot}  = \itFi \, \overset{4}{\mathbf{I}} \,\iFi
\end{equation}
After rearranging we have
\begin{equation}
    \left( \Stot
    - 2\pdv{\Psieq}{\Ctot} 
    - 2 \iFi \,\pdv{\Psineq}{\Ce} \,\itFi \right) : \frac{1}{2}\dot{\Ctot} 
    - \pdv{\Psineq}{\Ce}:\pdv{\Ce}{\Fi} : \dot{\Fi} \geq 0        
\end{equation}
This inequality holds for all values of \(\dot{\Ctot}\) and \(\dot{\Fi}\), which are independent of each other. This leads to the constitutive equation 
\begin{equation}
    \Stot = \Seq + \Sneq 
    = \underbrace{2 \pdv{\Psieq}{\Ctot}}_{\Seq}
    + \underbrace{2 \iFi \,\pdv{\Psineq}{\Ce} \,\itFi}_{\Sneq}
    = 2 \pdv{\Psitot}{\Ctot}
    \label{eq:stress_split}
\end{equation}
and the residual inequality
\begin{equation}
    \pdv{\Psineq}{\Ce}:\pdv{\Ce}{\Fi} : \dot{\Fi} \geq 0          
\end{equation}
which after simplification leads to 
\begin{equation}
    \pdv{\Psineq}{\Ce}: ({\mathbf{l}_{i}}^{T} \Ce + \Ce \mathbf{l}_{i}) \geq 0          
\end{equation}
and by exploiting symmetry properties we have
\begin{equation}
   2 \pdv{\Psineq}{\Ce}: ( \Ce \mathbf{l}_{i}) \geq 0     
   \label{eq:inequality_visc}
\end{equation}
where \(\li = \dot{\Fi}\iFi\) is termed the inelastic part of the spatial velocity gradient \(\ltot = \dot{\Ftot} \iF\). Furthermore, rewriting \cref{eq:inequality_visc} in the form
\begin{equation}
    2 \Fe \, \pdv{\Psineq}{\Ce}\, \tFe : (\itFe \Ce \,\li \,\iFe) \geq 0
\end{equation}
leads to 
\begin{equation}
    \tauneq \, \ibe : (\Fe \, \li \, \tFe) \geq 0
    \label{eq:inequality_viscous}
\end{equation}
where the relations 
\begin{equation}
    \tauneq = \Fe \,\Sneq \, \tFe = 2 \Fe \, \pdv{\Psineq}{\Ce}\, \tFe 
    \quad \text{and} \quad
    \be = \Fe \tFe
\end{equation}
have been used keeping \cref{eq:kirchoff_stress}, \cref{eq:secondpiolakirchoff_stress} \cref{eq:stress_split} and \cref{eq:def_right_left_cauchygreen} in mind.

The term \( \Fe \, \li \, \tFe\) can be split into a symmetric and skew-symmetric tensor as
\begin{equation}
    \Fe \, \li \, \tFe = \text{sym}(\Fe \, \li \, \tFe) + \text{skew}(\Fe \, \li \, \tFe)
    \label{eq:split_sym_skew_li}
\end{equation}
Considering the symmetric part, by virtue of \cref{eq:dfgrd_mul_split} and \cref{eq:rate_of_deformation} 
\begin{align}
    \nonumber \text{sym} (\Fe \, \li \, \tFe) 
    &= \frac{1}{2} \left(\Fe \, \li \, \tFe + {(\Fe \, \li \, \tFe)}^{T} \right)
    = \frac{1}{2} \left(\Fe  \left(\li + \li\right)  \tFe \right) \\
    \nonumber &= \frac{1}{2} \left(\Ftot \iFi  \left(2 \di \right)  \itFi \tF \right)
    = \frac{1}{2} \left(\Ftot \iFi  \left(\itFi \dot{\Ci} \iFi \right)  \itFi \tF \right) \\
    \nonumber&=  \frac{1}{2} \left(\Ftot (\iFi  \itFi) \dot{\Ci} (\iFi \itFi) \tF \right)= \frac{1}{2} \left(\Ftot \iCi \dot{\Ci} \iCi \tF \right)  \\
    \text{sym} (\Fe \, \li \, \tFe) 
    & = -\frac{1}{2} \left(\Ftot \dot{\overline{(\iCi)}} \tF \right)
\end{align}
where 
\begin{equation}
    \iCi \dot{\Ci} \iCi = - \dot{\overline{(\iCi)}}
\end{equation}
can be proved by taking the time derivative of the identitiy \(\Ci \iCi = \mathbf{I}\). Furthermore it can be shown that the inelastic right Cauchy-Green tensor
\begin{equation}
    \Ci = \tFi \Fi 
    = {(\iFe \Ftot)}^{T} (\iFe \Ftot) 
    = \tF (\itFe \iFe) \Ftot
    = \tF \ibe \Ftot
    \label{eq:inelastic_right_cauchygreen}
\end{equation}
and hence
\begin{equation}
    \be = \Ftot \iCi \tF
    \label{eq:be_Ci}
\end{equation}
The \cref{eq:split_sym_skew_li} thus reduces to 
\begin{equation}
    \Fe \, \li \, \tFe = -\frac{1}{2} \underbrace{\left(\Ftot \dot{\overline{(\iCi)}} \tF \right)}_{\mathcal{L}_{v} \be } + \text{skew}(\Fe \, \li \, \tFe)
\end{equation}
Here the notation \(\mathcal{L}_{v} \be \) stands for the Lie derivative of the contravariant tensor \(\be \). The Lie derivative has the meaning of pushing back a tensor quantity into the reference configuration, taking the time derivative and then pushing it forward into the current configuration. For a contravariant tensor \(\iF (\bullet) \itF \) represents the pull back transformation and \(\Ftot (\bullet) \tF \) represents the push forward transformation. Indeed we have by virtue of \cref{eq:be_Ci}
\begin{equation}
    \mathcal{L}_{v} \be 
    = \Ftot \dot{\overline{(\iF (\be) \itF)}} \tF
    = \Ftot \dot{\overline{(\iCi)}} \tF
\end{equation} 

Now assuming isotropy of the material and consequently \(\Psineq\) to be an isotropic tensor function. Therefore we may write \(\Psineq(\be)\) because both \(\Ce\) and \(\be\) are symmetric have the same eigen values. In the case \(\tauneq\) and \(\be\) commute, which gives symmetry of the tensor \(\tauneq \, \ibe\). Thus in \cref{eq:inequality_viscous} only the symmetric part of \( \Fe \, \li \, \tFe \) plays a role and gives rise to the (isotropic) inequality
\begin{equation}
    - \tauneq\,  \ibe : \frac{1}{2} (\mathcal{L}_{v} \be) \ibe \geq 0
    \label{eq:inequality_viscous_final}
\end{equation}
In order to fulfil \cref{eq:inequality_viscous_final} the evolution equation is chosen to be of the form
\begin{equation}
   -  \frac{1}{2} (\mathcal{L}_{v} \be) \ibe = \mathcal{V}^{-1} : \tauneq
   \label{eq:evolution_equation_1}
\end{equation}
where \(\mathcal{V}^{-1} = \mathcal{V}^{-1}(\be)\) is an isotropic rank four tensor which has to be positive definite and is given by
\begin{equation}
    \mathcal{V}^{-1} = \frac{1}{2\nd}\underbrace{\left( \overset{4}{\mathbf{I}} - \frac{1}{3} \mathbf{I} \otimes \mathbf{I}\right)}_{\substack{\text{Deviatoric} \\ \text{Map}}} + \frac{1}{3\nv} \underbrace{\left(  \frac{1}{3} \mathbf{I} \otimes \mathbf{I}\right)}_{\substack{\text{Volumetric} \\ \text{Map}}}
    \label{eq:isotropic_viscous_tensor}
\end{equation} 
Here, \(\nd > 0\) and \(\nv > 0\) represent the deviatoric and volumetric viscosities respectively and could possibly, but not necessarily, be deformation dependent.
If \cref{eq:isotropic_viscous_tensor} is inserted into \cref{eq:evolution_equation_1}, we finally obtain the evolution equation as
\begin{equation}
    -  (\mathcal{L}_{v} \be) \ibe 
    =  \frac{1}{\nd} \text{dev}(\tauneq) 
    + \frac{2}{3\nv} \text{vol}(\tauneq)
    \label{eq:evolution_equation_2}
\end{equation}
where 
\begin{equation}
    \text{vol}(\tauneq) = \frac{1}{3} \text{tr}(\tauneq)\, \mathbf{I} \quad \text{and} \quad
    \text{dev}(\tauneq) = \tauneq - \text{vol}(\tauneq)
\end{equation}


\subsection{Integration of the Evolution Equation}
The main objective in this section is to determine in an efficient manner the value of the governing inelastic internal variable 
\begin{align}
    \nonumber \be(\Ctot, \Fi) 
    &= \Fe \tFe = (\Ftot \iFi) (\Ftot \iFi)^{T}  = \Ftot (\iFi \itFi)  \tF \\
    \nonumber
    &= \Ftot (\iCi)  \tF \\
    \nonumber
    &= (\itF \Ctot) \iCi (\Ctot \iF)  \\
    \be(\Ctot, \Fi) 
    &= \itF (\Ctot \iFi \itFi \Ctot) \iF 
\end{align} for an advancement of time in order that the stresses may be evaluated when the material motion \(\Ftot\) is known, which is a typical setting in a finite element program.
For this, an operator split of the material time derivative of the \(\be\) is carried into an elastic predictor \(E\) and an inelastic correcter \(I\) given by
\begin{equation}
    \dot{\be} = \dot{\overline{\Ftot \, \iCi \, \tF}} 
    = \underbrace{\ltot\, \be + \be\, \tl}_{E}
    + \underbrace{\Ftot \dot{\overline{(\iCi)}} \tF}_{I}
\end{equation}

First, in the elastic predictor step the material time derivative of \(\iCi\) is set to zero and we have
\begin{equation}
    \tr{\iCi} = (\iCi)_{t=t_{n-1}} \rightarrow \tr{\be} 
    = (\Ftot)_{t=t_{n}} \, (\iCi)_{t=t_{n-1}} \,(\tF)_{t=t_{n}}
    \label{eq:be_tr}
\end{equation}
Then in the elastic corrector step, the spatial velocity gradient is set to zero, which leads to \(\mathcal{L}_{v} \be = \dot{\be}\) and finally to 
\begin{equation}
    \dot{\be} = -(2\mathcal{V}^{-1} : \tauneq) \, \be
\end{equation} 
Solving the above differential equation using the exponential mapping algorithm gives
\begin{equation}
    \be = \exp \left( -2 \int_{t_{n-1}}^{t}  \mathcal{V}^{-1} : \tauneq \odif{t} \right) \tr{\be}
\end{equation}
or by a first order accurate approximation gives
\begin{equation}
    \be \approxeq \exp \left( -2 \underbrace{(t_{n} - t_{n-1})}_{\Delta t}  {(\mathcal{V}^{-1} : \tauneq)}_{t=t_{n}}  \right) \tr{\be}
    \label{eq:evolution_equation_integrated_approx}
\end{equation}
Due to isotropy, \(\tauneq\) commutes with \(\be\) and also with \(\tr{\be}\). Since \(\mathcal{V}^{-1}\) is assumed to be isotropic, by virtue of \cref{eq:evolution_equation_1} and \cref{eq:evolution_equation_2} we can furthur write \cref{eq:evolution_equation_integrated_approx} as
\begin{equation}
    \prstretch{2}{i} = \exp \left( - \Delta t \left[ \frac{1}{\nd} \prtauneqdev + \frac{2}{9\nv} \text{tr}(\tauneq) \right] \right) \tr{\prstretch{2}{i}}
\end{equation}
where \(\text{dev}{(\tauneq)}_{i}\) are the principal values of \(\text{dev}{(\tauneq)}\), \(\prstretch{2}{i}\) the principal values of \({(\be)}_{t=t_{n}}\) and \(\tr{\prstretch{2}{i}}\) the principal values of \(\tr{\be}\). Furthermore taking the logarithm of both sides we have
\begin{equation}
    \prstrain{i} = - \Delta t \left( \frac{1}{2\nd} \prtauneqdev + \frac{1}{9\nv} \text{tr}(\tauneq) \right) + \tr{\prstrain{i}}
    \label{eq:principal_strain_neq}
\end{equation} 
where \(\prstrain{i} = \ln\prstretch{}{i}\) are the elastic logarithmic principal stretches.

Next, due to isotropy, we can consider the non-equilibrium part of the strain energy \(\Psineq\) as a function of the principal values \(\sqprstretch{i} = \prstretch{2}{i}\) of \(\be\) and we thus have
\begin{equation}
    \Psineq = \Psineq (\sqprstretch{1}, \sqprstretch{2}, \sqprstretch{3})
\end{equation}
which is futher split into volumetric and deviatoric parts given by
\begin{equation}
    \Psineq = ({\Psineq})_{D} (\sqprstretchdev{1}, \sqprstretchdev{2}, \sqprstretchdev{3}) + ({\Psineq})_{V} (\Je)
\end{equation}
where 
\begin{equation}
    \Je = \prstretch{}{1} \prstretch{}{2} \prstretch{}{3}
    \label{eq:jacobian_neq}
\end{equation}
denotes the determinant of \(\Fe\) and
\begin{equation}
    \sqprstretchdev{i} = \Je^{-2/3} \sqprstretch{i} = \Je^{-2/3} \prstretch{2}{i}
    \label{eq:principal_be_dev}
\end{equation}
the principal values of 
\begin{equation}
    \bebar = \Je^{-2/3} \, \be
\end{equation}
For the strain energy function \(\Psineq\) the Ogden class of models is used which after a split into deviatoric and volumetric parts is given by
\begin{equation}
    \Psineq = \sum_{r=1}^{N} \frac{\mumr}{\amr}
     \left(\sqprstretchdev{1}^{\amr/2} + \sqprstretchdev{2}^{\amr/2} + \sqprstretchdev{3}^{\amr/2}\right) 
     + \dfrac{\km}{4}\left( \Je^2 - 2 \ln\Je -1 \right)
     \label{eq:strain_energy_neq}
\end{equation}
where \((r = 1, \ldots , N)\) is the order of the Ogden model, \(\mum\) and \(\am\) are elasticity constants, such that for the shear modulus the relationship
\begin{equation}
    \mu_{vis} = \sum_{r=1}^{N} \frac{1}{2} \mumr \amr
\end{equation}
holds. Furthermore, the non-equilibrium part of the Kirchoff stress tensor \(\tauneq\) is then given by
\begin{equation}
    \tauneq = 2 \pdv{\Psineq}{\be}\, \be 
    = \sum_{i=1}^{3} \prtauneq \, \bm{n}_{i} \otimes \bm{n}_{i}
\end{equation}
The principal Kirchoff stresses are calculated by
\begin{equation}
    \prtauneq = \prtauneqdev +  \frac{1}{3} \text{tr}(\tauneq)
    \label{eq:principal_kirchoff_stress}
\end{equation}
where
\begin{equation}
    \text{dev}(\tauneq)_{i} 
    = \sum_{r=1}^{N} \mumr \left( \frac{2}{3} \,\sqprstretchdev{i}^{\amr/2} - \frac{1}{3} \,\sqprstretchdev{j}^{\amr/2} - \frac{1}{3} \,\sqprstretchdev{k}^{\amr/2} \right) \quad i \neq j \neq k 
    \label{eq:principal_kirchoff_stress_dev}
\end{equation}
and
\begin{equation}
    \frac{1}{3} \text{tr}(\tauneq) = \prtauneqvol
    = \frac{\km}{4}(\Je^2-1)
    \label{eq:principal_kirchoff_stress_vol}
\end{equation}
We thus have that \({\tauneq}_{i} = {\tauneq}_{i}(\prstrain{i})\). Thus  with \(\prstretch{}{i} = \exp(\prstrain{i})\), it is evident that \cref{eq:principal_strain_neq} is a non-linear equation in the elastic logarithmic strains \(\prstrain{i}\) which can be solved using the Newton-Raphson's method. 
The steps for the Newton-Raphson iterations are given below with \(i, j, k = 1,2,3\) and \(m\) is the current iteration number

\begin{enumerate}
    \item Develop the residual from the non-linear equation in \cref{eq:principal_strain_neq}
    \begin{equation}
       r_{i} = \prstrain{i} + \Delta t \left( \frac{1}{2\nd} \text{dev}{(\tauneq)}_{i} + \frac{1}{9\nv} \text{tr}(\tauneq) \right) - \tr{\prstrain{i}} = 0
       \label{eq:residual}
    \end{equation}
    \item Linearize the residual \cref{eq:residual}  around the logarithmic principal strain \(\prstrain{i} = (\prstrain{i})_{m}\) for the current iteration
    \begin{equation}
        r_{i} 
        \approxeq \underbrace{r_{i}|_{(\prstrain{i})_{m}}}_{(r_{i})_m} 
        + \underbrace{\left. \pdv{r_{i}}{\prstrain{j}} \right|_{(\prstrain{i})_{m}}}_{(K_{ij})_{m}} (\Delta \prstrain{j})_{m} = 0
        \label{eq:residual_linearized}
    \end{equation}
    \item Using \cref{eq:residual_linearized} solve for \((\Delta \prstrain{i})_{m}\)
    \begin{equation} 
        (K_{ij})_{m} (\Delta \prstrain{i})_{m} = -(r_{i})_{m}
        \label{eq:strain_increment_neq}
    \end{equation}
    where 
    \begin{equation}
        (K_{ij})_{m} = \delta_{ij} 
        + \Delta t \frac{1}{2\nd} \pdv{(\prtauneqdev)}{\prstrain{j}} 
        - \Delta t \frac{1}{3\nv} \pdv{(\frac{1}{3}\text{tr}(\tauneq))}{\prstrain{j}}
        \label{eq:K_newton_iteration}
    \end{equation}
    and with \(( i \neq j \neq k)\)
    \begin{align}
        \pdv{(\text{dev}(\tauneq)_{i})}{\prstrain{i}} 
        &= \sum_{r=1}^{N}\mumr \amr \left(
             \frac{4}{9} \,\sqprstretchdev{i}^{\amr/2}
            +\frac{1}{9} \,\sqprstretchdev{j}^{\amr/2}
            +\frac{1}{9} \,\sqprstretchdev{k}^{\amr/2} \right)  \label{eq:diff_prtauneqdevi_i}\\
        \pdv{(\text{dev}(\tauneq)_{i})}{\prstrain{j}} 
        &= \sum_{r=1}^{N}\mumr \amr \left(
            -\frac{2}{9} \,\sqprstretchdev{i}^{\amr/2}
            -\frac{2}{9} \,\sqprstretchdev{j}^{\amr/2}
            +\frac{1}{9} \,\sqprstretchdev{k}^{\amr/2} \right)  \label{eq:diff_prtauneqdevi_j}\\
        \pdv{(\frac{1}{3}\text{tr}(\tauneq))}{\prstrain{j}} 
        &= \km \Je^{2} \label{eq:diff_prtauneqvol}
    \end{align}
    \item Update the logarithmic principal strain for the next iteration and increment the iteration number
    \begin{equation}
        (\prstrain{i})_{m+1} = (\prstrain{i})_{m} + (\Delta \prstrain{i})_{m}
        \label{eq:strain_update_neq}
    \end{equation}
    and 
    \begin{equation}
        m = m+1
    \end{equation}
    \item Repeat steps 1.-4. until the norm of the residual
    \begin{equation}
        || \bm{r} || = \sqrt{\sum_{i=1}^{3} (r_{i})^{2}}
        \label{eq:residual_norm} 
    \end{equation}
    is less than the specified tolerance, i.e \(|| \bm{r} || < \text{tolerance}\)
\end{enumerate}

Keeping isotropy in mind \(\be\) and \(\tauneq\) have the same eigen vectors as \(\tr{\be}\), i.e \(\bm{n}_{i} = \tr{\bm{n}_{i}}\). Thus after convergence of the Newton-Raphsons iteration we can calculate
\begin{equation}
    \be 
    = \sum_{i=1}^{3} \prstretch{2}{i} \prvec{i} \otimes \prvec{i}
    = \sum_{i=1}^{3} \prstretch{2}{i} \tr{\prvec{i}} \otimes \tr{\prvec{i}} 
    \label{eq:def_be}
\end{equation}
and 
\begin{equation}
    \tauneq = 
    \sum_{i=1}^{3} \prtauneq \, \prvec{i} \otimes \prvec{i}
    = \sum_{i=1}^{3} \prtauneq \, \tr{\prvec{i}} \otimes \tr{\prvec{i}}
    \label{eq:tauneq} 
\end{equation}
where \((\tauneq)_i\) is given by \cref{eq:principal_kirchoff_stress}, \cref{eq:principal_kirchoff_stress_dev}, \cref{eq:principal_kirchoff_stress_vol} and \cref{eq:principal_be_dev}
\subsection{Solution of Weak Formulation: FE Procedure}
In a finite element program, the balance of mechanical energy \cref{eq:result_energy_balance} is solved in the weak form. In such a formulation, it is required to caclulate the stresses \(\tautot\) and the spatial tangent modulus \(\Cmattot\) given by the sum of their equilibrium and non-equilibrium parts as
\begin{align}
    \tautot &= \taueq + \tauneq \label{eq:tautot}\\
    \Cmattot &= \Cmateq + \Cmatneq \label{eq:Cmattot}
\end{align}
where \(\Cmattot\) is also called the 4-th order constitutive tensor. It can be calculated by pushing forward the material tangent modulus \(\Lmattot\) with \(\Ftot\) where
\begin{equation}
    \Lmattot = \Lmateq + \Lmatneq
\end{equation}
and keeping \cref{eq:stress_from_strain_energy}
\begin{equation}
    \Lmattot = \pdv{\Stot}{\Etot} = \pdv{\Psitot}{\Etot, \Etot}
\end{equation}

After integration of the evolution equation, \(\tauneq\) \cref{eq:tauneq} is already calculated. Thus it remains to calculate \(\taueq\), \(\Cmateq\) and \(\Cmatneq\)

\paragraph*{Calculation of Equilibrium Kirchoff Stress Tensor}
Let us again consider a strain energy function according to the Ogden model similar to that in \cref{eq:strain_energy_neq} for the equilibrium part \(\Psieq\) of \(\Psitot\) given by
\begin{equation}
    \Psieq = \sum_{r=1}^{N} \frac{\mur}{\ar}
     \left(\sqprstretchedev{1}^{\ar/2} + \sqprstretchedev{2}^{\ar/2} + \sqprstretchedev{3}^{\ar/2}\right) 
     + \dfrac{\K}{4}\left( \J^2 - 2 \ln\J -1 \right)
     \label{eq:strain_energy_eq}
\end{equation}
where
\begin{equation}
    \J = \prstretche{}{1} \prstretche{}{2} \prstretche{}{3}
\end{equation}
denotes the determinant of \(\Ftot\) and
\begin{equation}
    \sqprstretchedev{i} = \J^{-2/3} \sqprstretche{i} 
    = \J^{-2/3} \prstretche{2}{i}
    \label{eq:principal_b_dev}
\end{equation}
the principal values of 
\begin{equation}
    \bbar = \J^{-2/3} \, \btot = \sum_{i=1}^{3} \J^{-2/3} \sqprstretche{i} \, \eigvec{i} \otimes \eigvec{i} 
\end{equation}

The principal values of the equilibrium Kirchoff stress tensor
\begin{equation}
    \taueq = 
    \sum_{i=1}^{3} \prtaueq \, \prvec{i} \otimes \prvec{i}
    \label{eq:taueq} 
\end{equation}
considering isotropic elasticity \cite[see Sec 6.2.3]{Zienkiewicz2014} are given by
\begin{align}
    {\taueq}_{i} &= \pdv{\Psieq}{\prstraine{i}} \label{eq:principal_kirchoff_eq_1}\\
    \nonumber& = \pdv{\Psieq}{\sqprstretchedev{k}} \pdv{\sqprstretchedev{k}}{\prstretche{}{j}} \pdv{\prstretche{}{j}}{\prstraine{i}} 
    + \pdv{\Psieq}{\J} \pdv{\J}{\prstretche{}{j}} \pdv{\prstretche{}{j}}{\prstraine{i}} \\
    {\taueq}_{i} &= \underbrace{\pdv{\Psieq}{\sqprstretchedev{j}} \pdv{\sqprstretchedev{j}}{\prstretche{}{i}} \prstretche{}{i}}_{\text{dev}{(\taueq)}_{i}}
    + \underbrace{\pdv{\Psieq}{\J} \J}_{\text{vol}{(\taueq)}_{i}} \label{eq:principal_kirchoff_eq_2}
\end{align}
where
\begin{equation}
    \text{dev}{(\taueq)_{i}} = \sum_{r=1}^{N} \mur \left( \frac{2}{3} \,\sqprstretchedev{i}^{\ar/2} - \frac{1}{3} \,\sqprstretchedev{j}^{\ar/2} - \frac{1}{3} \,\sqprstretchedev{k}^{\ar/2} \right) \quad i \neq j \neq k 
    \label{eq:principal_kirchoff_stress_eq_dev}
\end{equation}
and 
\begin{equation}
    \text{vol}{(\taueq)_{i}} = \frac{1}{3} \text{tr}(\taueq) 
    = \frac{\K}{4}(\J^2-1)
    \label{eq:principal_kirchoff_stress_eq_vol}
\end{equation}


\paragraph*{Calculation of Equilibrium Spatial Tangent Stiffness Modulus} Once the principal values of the equilibrium Kirchoff stress tensor are known, it remains to calculate the equilibrium spatial tangent stiffness modulus \cite[Eq 6.50]{Zienkiewicz2014} given by
\begin{align}
    \nonumber\J \Cmateq 
    &= 
   \sum_{i=1}^{3} \sum_{j=1}^{3}\left( c_{ij} - 2 \prtaueq \delta_{ij} \right) \eigvec{i} \otimes \eigvec{i} \otimes \eigvec{j} \otimes \eigvec{j} \\
    &+ \frac{1}{2} \sum_{i=1}^{3} \sum_{j\neq i=1}^{3} \left( g_{ij} \right) \,  \left[\eigvec{i} \otimes \eigvec{j} \otimes \eigvec{i} \otimes \eigvec{j} + \eigvec{i} \otimes \eigvec{j} \otimes \eigvec{j} \otimes \eigvec{i}\right] \label{eq:Cmat_eq} 
\end{align}
In the above equation, keeping \cref{eq:principal_kirchoff_eq_1} in mind we have
\begin{align}
    \nonumber c_{ij} 
    &= \pdv{\Psieq}{\prstraine{i}, \prstraine{j}} 
    = \pdv*{\pdv{\Psieq}{\prstraine{j}}}{\prstraine{i}} = \pdv{\prtaueqsub{j}}{\prstraine{i}}  \\
    c_{ij}  &=  \pdv{(\prtaueqdevsub{j})}{\prstraine{i}} + \pdv{(\prtaueqvol)}{\prstraine{i}}
    \label{eq:c_ij}
\end{align}
where similar to \cref{eq:diff_prtauneqdevi_i}, \cref{eq:diff_prtauneqdevi_j}, \cref{eq:diff_prtaueqvol}, we have 
\begin{align}
    \pdv{(\prtaueqdev)}{\prstraine{i}} 
    &= \sum_{r=1}^{N}\mur \ar \left(
         \frac{4}{9} \,\sqprstretchedev{i}^{\ar/2}
        +\frac{1}{9} \,\sqprstretchedev{j}^{\ar/2}
        +\frac{1}{9} \,\sqprstretchedev{k}^{\ar/2} \right) \label{eq:diff_prtaueqdevi_i} \\
    \pdv{(\prtaueqdev)}{\prstraine{j}} 
    &= \sum_{r=1}^{N}\mur \ar \left(
        -\frac{2}{9} \,\sqprstretchedev{i}^{\ar/2}
        -\frac{2}{9} \,\sqprstretchedev{j}^{\ar/2}
        +\frac{1}{9} \,\sqprstretchedev{k}^{\ar/2} \right)\label{eq:diff_prtaueqdevi_j} \\
    \pdv{(\prtaueqvol)}{\prstraine{j}}  
    &= \pdv{(\frac{1}{3}\text{tr}(\taueq))}{\prstraine{j}} 
    = \K \J^{2} \quad i,j = 1,2,3 \label{eq:diff_prtaueqvol}
\end{align}
with \(k \neq j \neq i = 1,2,3\) in \cref{eq:diff_prtauneqdevi_i} \cref{eq:diff_prtauneqdevi_j} and 
\begin{equation}
    g_{ij} = \begin{cases}
        \begin{aligned}
            \dfrac{\prtaueqsub{i} \prstretche{2}{j} 
            -  \prtaueqsub{j} \prstretche{2}{i}}
            {\prstretche{2}{i} 
            -  \prstretche{2}{j}},              
            && \text{if } \prstretche{}{i} \neq \prstretche{}{j}\\
            \pdv{(\prtaueqsub{i}- \prtaueqsub{j})}{\prstraine{i}},              
            && \text{if } \prstretche{}{i} = \prstretche{}{j}
        \end{aligned}
    \end{cases}
    \quad i,j=1,2,3
    \label{eq:g_ij}
\end{equation}
For the second case where the equilibrium principal stretches are equal, considering \cref{eq:principal_kirchoff_eq_2}, \cref{eq:principal_kirchoff_stress_eq_dev} and \cref{eq:principal_kirchoff_stress_eq_vol}, we first evaluate
\begin{equation}
    \prtaueqsub{i} - \prtaueqsub{j}
    = \sum_{r=1}^{N} \mur \left( \sqprstretchedev{i}^{\ar/2} - \sqprstretchedev{j}^{\ar/2} \right)
\end{equation}
which after differentiation with \(\prstraine{i}\) gives
\begin{equation}
    \pdv{(\prtaueqsub{i}- \prtaueqsub{j})}{\prstraine{i}} 
    = \sum_{r=1}^{N} \mur \ar 
    \left( \frac{2}{3}\sqprstretchedev{i}^{\ar/2} 
    + \frac{1}{3}\sqprstretchedev{j}^{\ar/2} \right) 
    \label{eq:g_ij_equalstretch}
\end{equation}

\paragraph*{Calculation of Non-Equilibrium Spatial Tangent Stiffness Modulus}
For the non-equilibrium part, instead of calculating the spatial tangent stiffness modulus \(\Cmatneq\) by push forward of the material tangent stiffness modulus \(\Lmatneq\) with \(\Ftot\),  an alternative method is used \cite{Reese1998Sep}. Here, instead of the expression in \cref{eq:visc_simple_case}, an equivalent alternative multiplicative split of \(\Ftot\) is considered
\begin{equation}
    \Ftot = \tr{\Fe} (\Fi)_{t=t_{n-1}}
\end{equation}
which follows from the fact that \(\tr{\Fe} = (\Ftot)_{t=t_{n}} (\iFi)_{t=t_{n-1}}\). Due to this, at time \(t = t_{n}\), \((\iFi)_{t=t_{n-1}}\) has to be treated constant. Thus, the increment
\begin{equation}
    \Delta \Ftot = \Delta \tr{\Fe} (\Fi)_{t=t_{n-1}}
\end{equation}
depends only on the increment \(\Delta \tr{\Fe}\). Hence, an intermediate second Piola-Kirchoff stress tensor \(\Sneqtil\) is sought. Consider the non-equilibrium second Piola-Kirchoff stress tensor given by
\begin{align}
    \nonumber \Sneq &= \iF \tauneq \itF \\
    &=  (\iFi)_{t=t_{n-1}} \underbrace{\tr{\iFe} \tauneq \tr{\itFe}}_{\Sneqtil} (\itFi)_{t=t_{n-1}} 
\end{align}
which leads to the definition
\begin{equation}
    \Sneqtil = (\Fi)_{t=t_{n-1}} \Sneq (\tFi)_{t=t_{n-1}}
\end{equation}
The non-equilibrium spatial tangent modulus \(\Cmatneq\) is then found by first determining the material tensor
\begin{equation}
    \Lmatneqtil = 2 \pdv{\Sneqtil}{\tr{\Ce}}
\end{equation}
and then pushing it forward with \(\tr{\Fe}\). Using the spectral decomposition, we also have 
\begin{equation}
    \Sneqtil = \sum_{i=1}^{3} \frac{\prtauneq}{\tr{\prstretch{2}{i}}} \Eigvectil{i} \otimes \Eigvectil{i}
\end{equation}
where 
\begin{equation}
    \prSneqtil = \frac{\prtauneq}{\tr{\prstretch{2}{i}}}.
\end{equation}
Finally, taking into account
\begin{equation}
    \Delta \Sneqtil = \Lmatneqtil : \frac{1}{2} \Delta \tr{\Ce}
\end{equation}
leads to the non-equilibrium tangent stiffness modulus in the intermediate configuration given by
\begin{align}
  \nonumber  \Lmatneqtil 
&= \sum_{i=1}^{3} \sum_{j=1}^{3} \left( L_{ij} \right) \Eigvectil{i}\otimes \Eigvectil{i} \otimes \Eigvectil{j} \otimes \Eigvectil{j} \\
&+\sum_{i=1}^{3} \sum_{j\neq i=1}^{3} \left(G_{ij} \right) \left[\Eigvectil{i}\otimes \Eigvectil{j} \otimes \Eigvectil{i} \otimes \Eigvectil{j} + \Eigvectil{i}\otimes \Eigvectil{j} \otimes \Eigvectil{j} \otimes \Eigvectil{i} \right]\\
\Lmatneqtil&= \sum_{i=1}^{3}\sum_{j=1}^{3}\sum_{k=1}^{3}\sum_{l=1}^{3} (\tilde{L}_{ijkl}) \Eigvectil{i} \otimes \Eigvectil{j}  \otimes \Eigvectil{k}\otimes \Eigvectil{l}
\label{eq:Lneq_til}
\end{align}
where 
\begin{align}
    L_{ij} 
    &= \frac{C^{\text{alg}}_{ij} - 2 \prtauneq \delta_{ij}}{\tr{\prstretch{2}{i}}\tr{\prstretch{2}{j}}} \quad i,j=1,2,3 \label{eq:Lij_neq}\\[0.5em]
    G_{ij} 
    &= \begin{cases}
        \begin{aligned}
            &\frac{\prSneqsub{j}- \prSneqsub{i}}{\tr{\prstretch{2}{j}} - \tr{\prstretch{2}{i}}},  && \text{if } \tr{\prstretch{}{i}} \neq \tr{\prstretch{}{j}} \\[0.5em]
            &\pdv{\left(\prSneqsub{j}- \prSneqsub{i}\right)}{\tr{\prstretch{2}{j}}}, && \text{if } \tr{\prstretch{}{i}} = \tr{\prstretch{}{j}} 
        \end{aligned}
        \quad i,j=1,2,3
    \end{cases}
    \label{eq:Gij_neq}
\end{align}
Here again, in \cref{eq:Gij_neq} for the case of coalesence of the trial principal stretches, i.e. for \(\tr{\prstretch{}{i}} \to \tr{\prstretch{}{j}}\), the limiting value is calculated using L'Hospital rule. This expression can then be calculated using \cref{eq:Lij_neq} \cite[cf.][Eq 3.263]{Wriggers} and is given by 
\begin{equation}
\pdv{\left(\prSneqsub{j}- \prSneqsub{i}\right)}{\tr{\prstretch{2}{j}}} = \dfrac{1}{2} L_{ii} - L_{ij}
\end{equation}
Furthermore, in \cref{eq:Lij_neq} we have for the algorithmic tangent matrix 
\begin{align}
    \nonumber C^{\text{alg}}_{ij} 
    &= \pdv{\prtaueq}{\prstraine{k}} \K^{-1}_{kj}\\[0.5em]
    &=  \left[ \pdv{(\prtauneqdevsub{j})}{\prstrain{k}} + \pdv{(\prtauneqvol)}{\prstrain{k}} \right] \K^{-1}_{kj}
    \label{eq:C_alg}
\end{align}
which can be found by using the values from \cref{eq:diff_prtauneqdevi_i}, \cref{eq:diff_prtauneqdevi_j}, \cref{eq:diff_prtaueqvol} and \cref{eq:K_newton_iteration} after the convergence of the Newton-Raphson iteration procedure.
Finally after push forward with \(\tr{\Fe}\), \cref{eq:Lneq_til} yields
\begin{equation}
    \Cmatneq = \sum_{i=1}^{3}\sum_{j=1}^{3}\sum_{k=1}^{3}\sum_{l=1}^{3} (\tilde{L}_{ijkl}) \tr{\prstretch{}{i}} \tr{\prstretch{}{j}} \tr{\prstretch{}{k}} \tr{\prstretch{}{l}} \eigvec{i} \otimes \eigvec{j}  \otimes \eigvec{k}\otimes \eigvec{l}
    \label{eq:Cmatneq}
\end{equation}

\subsection{Jaumann Rate Formulation of Stiffness Modulus}
In the finite element formulation for large deformations, the construction of a rate form for constitutive equations deduced from a strain energy function is required. This is easily performed in the reference configuration and takes the form
\begin{equation}
    \dot{\Stot} = \Lmattot : \dot{\Etot}
    \label{eq:constitutive_rate_form_material}
\end{equation}
However such a definition is not appropriate for the rate of Cauchy stress tensor \(\dot{\bm{\sigma}}\) or the Kirchoff stress tensor \(\dot{\tautot}\), since they are related to different configurations at time \(t_{n}\) and \(t_{n-1}\) and thus would not satisfy the requirements of material objectivity. 

In order to compute an objective time derivative we consider with the help of \cref{eq:secondpiolakirchoff_stress}, \cref{eq:spatVelGradient} 
\begin{align}
    \nonumber \dot{\tautot} 
    &= \dot{\overline{(\Ftot \Stot \tF)}} \\
    \nonumber
    &= \dot{\Ftot} \Stot \tF + \Ftot \dot{\Stot} \tF + \Ftot \Stot \dot{\tF} \\
    \nonumber
     &= \ltot \Ftot \Stot \tF + \Ftot \dot{\Stot} \tF + \Ftot \Stot \tF \tl \\
     \dot{\tautot}
     &= \ltot \tautot + \Ftot \dot{\Stot} \tF + \tautot \tl
     \label{eq:trusdell_rate_kirchoff_stress}
\end{align}
which leads to the definition of the Trusdell rate~\cite{Truesdell1965} or equivalently the Lie derivative of the Kirchoff stress tensor \(\taudottruss\) given by
\begin{equation}
    \taudottruss = \mathcal{L}_v \tautot 
    =  \Ftot \dot{\Stot} \tF
    = \dot{\tautot} - \ltot \tautot - \tautot \tl
    \label{eq:trusdell_rate_kirchoff_stress}
\end{equation}
Another objective rate form of the Kirchoff stress tensor, which is widely used in finite element formulations and also ABAQUS, is the Jaumann rate which is given by
\begin{equation}
    \taudotjaumann =  \dot{\tautot} - \wtot \tautot - \tautot \tw 
    \label{eq:jaumann_rate_kirchoff_stress}
\end{equation}
where \(\wtot\) is the spin tensor. Keeping in mind \cref{spatial_velocity_gradient}, it can be shown that
\begin{equation}
    \taudotjaumann = \dot{\tautot} - \ltot \tautot - \tautot \tl + \dtot \tautot + \tautot \td
\end{equation}
Comparing \cref{eq:trusdell_rate_kirchoff_stress} with the above equation and by virtue of symmetry of the deformation tensor \(\dtot\), we have 
\begin{equation}
    \taudottruss = \taudotjaumann - \dtot \tautot - \tautot \dtot 
    \label{eq:trusdell_jaumann_rate_relation}   
\end{equation}
which gives a relation between the Trussdell rate and Jaumann rate of the Kirchoff stress tensor.

With the help of \cref{eq:trusdell_rate_kirchoff_stress}, \cref{eq:constitutive_rate_form_material} and \cref{eq:rate_of_deformation}, we have
\begin{align}
    \nonumber \taudottruss 
    \nonumber &= \Ftot \dot{\Stot} \tF \\
    \nonumber &= \Ftot \left(\Lmattot : \tF \dtot \Ftot \right) \tF \\ 
    \nonumber &= \Ftot \Ftot (\Lmattot) \tF \tF : \dtot \\
    \taudottruss &= \Cmattot : \dtot
    \label{eq:constitutive_rate_form_spatial}
\end{align}
where \cref{eq:constitutive_rate_form_spatial} gives the objective (Trussdell) rate formulation for the constitutive relation in the current configuration. Furthermore, using \cref{eq:trusdell_jaumann_rate_relation} in \cref{eq:constitutive_rate_form_spatial}, we have 
\begin{equation}
    \taudotjaumann = \Cmattot : \dtot  + \dtot \tautot + \tautot \dtot
\end{equation}
which finally leads to the Jaumann rate formulation of the constitutive relation
\begin{equation}
    \taudotjaumann = {\Cmattot}{\,\big|}^{\text{FE}}_{\tautot} : \dtot
\end{equation}
where
\begin{equation}
    {\Cmattot}{\,\big|}^{\text{FE}}_{\tautot} = {\Cmattot} + 
    \frac{1}{2}\left[ \delta_{ik} {\tau}_{jl} + \delta_{il} {\tau}_{jk} + \delta_{jl} {\tau}_{ik} + \delta_{jk} {\tau}_{il} \right] \evec{i} \otimes \evec{j} \otimes \evec{k} \otimes \evec{l}
\end{equation}
In the above equation, the first term when contracted with \(\dtot\) contributes to the objective rate of the Kirchoff stress tensor while the second term when contracted with \(\dtot\) contributes to the rate of the Kirchoff stress tensor associated with local spin or rigid body motion. Furthermore, the rate formulation of the constitutive relation involving the Cauchy stress tensor is given by 
\begin{equation}
    \dot{\bm{\sigma}} = {\Cmattot}{\,\big|}^{\text{FE}}_{\bm{\sigma}} : \dtot
    = \frac{1}{J} {\Cmattot}{\,\big|}^{\text{FE}}_{\tautot}  : \dtot
\end{equation}
where
\begin{align}
    {\Cmattot}{\,\big|}^{\text{FE}}_{\bm{\sigma}}
    &= \frac{1}{J}{\Cmattot} + 
    \frac{1}{2J}\left[ \delta_{ik} {\tau}_{jl} + \delta_{il} {\tau}_{jk} + \delta_{jl} {\tau}_{ik} + \delta_{jk} {\tau}_{il} \right] \evec{i} \otimes \evec{j} \otimes \evec{k} \otimes \evec{l}\\
    &=\sum_{i=1}^{3}\sum_{j=1}^{3}\sum_{k=1}^{3}\sum_{l=1}^{3} ({C}_{ijkl}) \evec{i} \otimes \evec{j}  \otimes \evec{k}\otimes \evec{l}
    \label{eq:jaumann_spatial_tangent_stiffness}
\end{align}
is the spatial tangent stiffness modulus which is symmetric \cite{Reese1998Sep}. Here, the coefficient matrix \(({C}_{ijkl})\) is sparse, and thus, can be stored in the Voigt notation form given by 
\begin{equation}
    \left({\Cmattot}{\,\big|}^{\text{FE}}_{\bm{\sigma}}\right)_{\text{VOIGT}} = 
    C_{ij}^{\text{ABAQUS}} = \left[\begin{array}{cccccc}
        C_{1111} & C_{1122} & C_{1133} & C_{1112} & C_{1113} & C_{1123} \\
        { } & C_{2222} & C_{2233} & C_{2212} & C_{2213} & C_{2223} \\
         &  & C_{3333} & C_{3312} & C_{3313} & C_{3323} \\
         &  &  & C_{1212} & C_{1213} & C_{1223} \\
         \multicolumn{4}{c}{\text{sym.}} & C_{1313} & C_{1323} \\
         &  &  & { } & { } & C_{2323} 
    \end{array} \right]
    \label{eq:abaqus_tangent_stiffness_modulus}
\end{equation} 

and is also implemented in the ABAQUS finite element software \cite{Abaqus614}. Lastly, for the finite viscoelasticity theory introduced earlier, we have for the equilibrium and non-equilibrium parts
\begin{align}
    \nonumber {\Cmateq}{\,\big|}^{\text{FE}}_{\bm{\sigma}} 
    &= \frac{1}{J}{\Cmateq}
   + \frac{1}{2J} \left[ \delta_{ik} {\tau_{\text{EQ}}}_{jl} + \delta_{il} {\tau_{\text{EQ}}}_{jk} \right. \\
    &\hspace{3cm} \left. + \delta_{jl} {\tau_{\text{EQ}}}_{ik} + \delta_{jk} {\tau_{\text{EQ}}}_{il} \right]  \evec{i} \otimes \evec{j} \otimes \evec{k} \otimes \evec{l}
   \label{eq:jaumann_spatial_tangent_stiffness_eq}
\end{align}
and 
\begin{align}
    \nonumber {\Cmatneq}{\,\big|}^{\text{FE}}_{\bm{\sigma}}
    &= \frac{1}{J}{\Cmatneq}
   + \frac{1}{2J} \left[ \delta_{ik} {\tau_{\text{NEQ}}}_{jl} + \delta_{il} {\tau_{\text{NEQ}}}_{jk} \right. \\
    &\hspace{3cm} \left. + \delta_{jl} {\tau_{\text{NEQ}}}_{ik} + \delta_{jk} {\tau_{\text{NEQ}}}_{il} \right]  \evec{i} \otimes \evec{j} \otimes \evec{k} \otimes \evec{l}
    \label{eq:jaumann_spatial_tangent_stiffness_neq}
 \end{align}

